\documentclass[t]{beamer}
\usepackage{helvet}
\usepackage{calc}
\usepackage[utf8]{inputenc}
\usepackage[english]{babel}

\usetheme{Ilmenau}

\setbeamercovered{transparent}
\setbeamertemplate{navigation symbols}{}

\usepackage{units}
\usepackage{amsbsy}
\usepackage{amsmath}
\usepackage{amssymb}
\usepackage{graphics}
\usepackage{graphicx}
\usepackage{epsf}
\usepackage{epsfig}
\usepackage{fixmath}
%\usepackage{pgfmath}
\usepackage{wrapfig}


\title{Gitify your life}
\subtitle{web, blog, configs, data, and backups}
\author{Richard Hartmann,\\
RichiH@\{freenode,OFTC,IRCnet\},\\
richih.mailinglist@gmail.com}
\date{2013-03-10}



\begin{document}

% hide all subsections
\setcounter{tocdepth}{1}

\begin{frame}
	\titlepage
\end{frame}

\begin{frame}
	\frametitle{Outline}
	\tableofcontents
\end{frame}


\section{Intro}

\begin{frame}
	\frametitle{Outline}
	\tableofcontents[currentsection]
\end{frame}

\subsection{Personal stuff}

\begin{frame}
	\frametitle{Who am I?}
	\begin{itemize}
		\item Richard "RichiH" Hartmann
		\item Backbone and project manager at Globalways AG
		\item freenode \& OFTC staff
		\item Author of vcsh
		\item Passionate about FLOSS
	\end{itemize}
\end{frame}

\subsection{The basics}

\begin{frame}
	\frametitle{What is Git?}
	\begin{itemize}
		\item Version control system
		\item Distributed
		\begin{itemize}
			\item No need for central repository
			\item Allows you to commit while offline
		\end{itemize}
		\item Stores commits (parent commit reference, commit message, root tree object) and tree objects (blobs and other tree objects)
		\item Light-weight branches
		\item pre-/post-action hooks
		\item Full history in every checkout
	\end{itemize}
\end{frame}


\section{ikiwiki}

\begin{frame}
	\frametitle{Outline}
	\tableofcontents[currentsection]
\end{frame}

\begin{frame}
		\begin{center}
			\vfill
			\vfill
			\textbf{ikiwiki}
			\vfill
			\textit{ikiwiki is a wiki compiler. It converts wiki pages into HTML pages suitable for publishing on a website}
			\vfill
			\vfill
		\end{center}
\end{frame}

\subsection{Background}

\begin{frame}
	\frametitle{What is ikiwiki?}
	\begin{itemize}
		\item Written in Perl
		\item Supported back-ends: Git, Bazaar, Darcs, GNU Arch, Mercurial, monotone, and Subversion
		\item Parses various markup languages
		\item Offers different ways of editing content
		\item Extensive templating and CSS support
		\item Acts as Wiki, CMS, and blog
		\item RSS and Atom feed for whole site, per page, per tag, etc
		\item Supports OpenID
	\end{itemize}
\end{frame}

\begin{frame}
	\frametitle{Supported markup languages}
	\begin{itemize}
			% TODO are wikilink and directives exlcisive to markdown?
		\item MarkDown
		\begin{itemize}
			\item WikiLink (\texttt{[[LinkToArticle]]})
			\item directives, e.g.
			\begin{itemize}
				\item \texttt{[[!tag talk/gitify]]}
				\item \texttt{[[!author RichiH]]}
				\item etc
			\end{itemize}
		\end{itemize}
		\item WikiText
		\item reStructuredText
		\item Textile
		\item plain HTML
	\end{itemize}
\end{frame}

\begin{frame}
	\frametitle{How does it work?}
	\begin{itemize}
		\item User edits web page or commits and pushes source files
		\item Partial/full rebuild triggered by cgi or commit hook
		\item Parses input files
		\item Compiles into HTML, create new pages, updates RSS, etc
		\item Commits MarkDown source for autocreated/-changed pages into repository
		\item User can optionally pull changes to local repository
	\end{itemize}
\end{frame}

\subsection{Use cases}

\begin{frame}
	\frametitle{Common uses}
	\begin{itemize}
		\item Public Wiki
		\item Private notes
		\item Blog
		\item CMS
	\end{itemize}
\end{frame}

\begin{frame}
	\frametitle{Adding/editing content}
	\begin{itemize}
		\item Web-based text editing (useful, but boring)
		\item Web-based WYSIWYG (via plugins/wmd)
		\item CLI-based (awesome!)
	\end{itemize}
\end{frame}

\begin{frame}
	\frametitle{Advanced usage}
	\begin{itemize}
		\item Interface with source files, only
		\item Maintain wiki and docs in the same repository as your source code
		\item Separate staging or even preview branches with output into different directories
	\end{itemize}
\end{frame}


\section{etckeeper}

\begin{frame}
	\frametitle{Outline}
	\tableofcontents[currentsection]
\end{frame}

\begin{frame}
		\begin{center}
			\vfill
			\vfill
			\textbf{etckeeper}
			\vfill
			\textit{etckeeper is a collection of tools to let /etc be stored in a Git, Mercurial, Darcs, or Bazaar repository}
			\vfill
			\vfill
		\end{center}
\end{frame}

\subsection{Minimal, quick overview}

\begin{frame}
	\frametitle{In a word}
	\begin{itemize}
		\item Implemented in POSIX shell
		\item Auto-commits /etc prior to and after all actions by package manager
		\item Hooks into apt, yum, pacman-g2, and cron
		\item Allows manual commits
		\item Various back-ends
		\begin{itemize}
			\item Bazaar
			\item Darcs
			\item Git
			\item Mercurial
		\end{itemize}
		\item Easy way to recover from failures, misconfiguration or to clone machines
	\end{itemize}
\end{frame}


\section{vcsh}
\begin{frame}
	\frametitle{Outline}
	\tableofcontents[currentsection]
\end{frame}

\begin{frame}
		\begin{center}
			\vfill
			\vfill
			\textbf{vcsh}
			\vfill
			\textit{manage config files in \$HOME via fake bare Git repositories}
			\vfill
			\vfill
		\end{center}
\end{frame}

\subsection{Technical details}

\begin{frame}
	\frametitle{What is vcsh?}
	\begin{itemize}
		\item Implemented in POSIX shell
		\item "version control shell" or "version control system \texttt{\$HOME}"
		\item Based on Git, but...
		\begin{itemize}
			\item Git is unable to maintain several working copies in one directory
			\item Sucks if you want to keep your configs in Git
		\end{itemize}
		\item vcsh uses fake bare Git repositories to work around this
		\item Powerful and extensible hook system
		\item Think of it as an extension to Git
	\end{itemize}
\end{frame}

\begin{frame}
	\frametitle{fake bare.. what?}
	\begin{itemize}
		\item Normal Git repository:
		\begin{itemize}
			\item working copy in \texttt{\$GIT\_WORK\_TREE}
			\item Git data in \texttt{\$GIT\_WORK\_TREE/.git} aka \texttt{\$GIT\_DIR}
		\end{itemize}
		\item Bare Git repository:
		\begin{itemize}
			\item Git data in \texttt{\$GIT\_DIR}
			\item no  \texttt{\$GIT\_WORK\_TREE}
		\end{itemize}
		\item Fake bare Git repository:
		\begin{itemize}
			\item working copy in \texttt{\$GIT\_WORK\_TREE}
			\item Git data in \texttt{\$GIT\_DIR}
			\item \texttt{\$GIT\_WORK\_TREE == \$HOME}
			\item \texttt{\$GIT\_DIR == \$XDG\_CONFIG\_HOME/vcsh/repo.d/\$repo.vcsh}
			\item \texttt{core.bare = false}
		\end{itemize}
	\end{itemize}
\end{frame}

\begin{frame}
	\frametitle{Problems with fake bare Git repos}
	\begin{itemize}
		\item Fake bare repos are messy to set up and use, but very easy to get wrong
		\item Reason why Git disallows shared \texttt{\$GIT\_WORK\_TREE}: complexity due to context-dependency
		\item Mistakes lead to confusion and/or data loss; imagine \texttt{\$GIT\_WORK\_TREE} set and
		\begin{itemize}
			\item \texttt{git add}
			\item \texttt{git reset --hard HEAD\~{}1}
			\item \texttt{git checkout -- *}
			\item \texttt{git clean -f}
		\end{itemize}
	\end{itemize}
\end{frame}

\begin{frame}
	\frametitle{Solution: vcsh}
	\begin{itemize}
		\item Wraps around Git
		\item Hides complexity and does sanity checks
		\item Several Git repositories checked out into \texttt{\$HOME} at once
		\begin{itemize}
			\item One repository for Zsh, Vim, mplayer, etc
			\item Allows specific subsets of repositories per host
		\end{itemize}
		\item Manages complete repository life-cycle
	\end{itemize}
\end{frame}

\subsection{Using vcsh}

\begin{frame}
	\frametitle{Create new repository}
	\texttt{ \\
		\# Create new repository \\
		vcsh init vim \\
		\# Add files to it \\
		vcsh run vim git add .vim .vimrc \\
		\# Commit using shorthand form \\
		vcsh vim commit \\
		\# Push using longhand form \\
		vcsh run vim git push
	}
\end{frame}

\begin{frame}
	\frametitle{Made-up life-cycle}
	\texttt{ \\
		\# Clone repository into new name zsh \\
		vcsh clone git://github.com/RichiH/zshrc.git zsh \\
		\# Display all files managed by this repository \\
		vcsh run zsh git ls-files \\
		\# Rename repository just because \\
		vcsh rename zsh zshrc \\
		\# Delete repository \\
		vcsh delete zshrc
	}
\end{frame}

\begin{frame}
	\frametitle{run vs enter}
	\texttt{ \\
		\# Do everything from outside \\
		vcsh run zsh git add .zshrc \\
		vcsh run zsh git commit \\
		vcsh run zsh git push \\
		\# The same commands, but from within\\
		vcsh enter zsh \\
		git add .zshrc \\
		git commit \\
		git push \\
		exit
	}
\end{frame}

\begin{frame}
	\frametitle{Advanced usage}
	\begin{itemize}
		\item Have your prompt display vcsh information
		\item git-annex within vcsh to manage non-configuration files in \texttt{\$HOME}
		\item Floating backups in arbitrary working copies
		\begin{itemize}
			\item \texttt{.git/}
			\item Working copy
			\item Complete repository, including objects, etc
		\end{itemize}
		\item Use Git on top of or in parallel to other VCSs
	\end{itemize}
\end{frame}


\section{git-annex}

\begin{frame}
	\frametitle{Outline}
	\tableofcontents[currentsection]
\end{frame}

\begin{frame}
		\begin{center}
			\vfill
			\vfill
			\textbf{git-annex}
			\vfill
			\textit{manage files with Git, without checking their contents in}
			\vfill
			\vfill
		\end{center}
\end{frame}

\subsection{Background}

\begin{frame}
	\frametitle{What is git-annex?}
	\begin{itemize}
		\item Based on Git
		\item Maintains metadata in Git, actual files in the annex
		\item Still allows you to check files into Git if you want to
		\item Written with low bandwidth and flaky connections in mind
		\item Various work-flows (more on that soon)
	\end{itemize}
\end{frame}

\begin{frame}
	\frametitle{Internal workings 1/2}
	\begin{itemize}
		\item Written in Haskell, so strong typing etc, internally
		\item Uses rsync to transfer data
		\item Indirect mode
		\begin{itemize}
			\item Moves files into \texttt{.git/annex/objects}
			\item Makes them read-only
			\item Replaces them with symlinks
			\item Forces you to \texttt{git annex unlock} and \texttt{git annex add}, leading to concious decisions about changes
		\end{itemize}
		\item Can either discard or keep old data, depending on setup
	\end{itemize}
\end{frame}

\begin{frame}
	\frametitle{Internal workings 2/2}
	\begin{itemize}
		\item Uses UUIDs to identify each repository
		\item Stores tracking information in \texttt{git-annex} branch
		\item Gives every single repository full information about all files
		\item Tracking information designed to work with union merge: \\
			{\footnotesize \texttt{1361402708.089154s 1 0d39904f-de8d-1638-92af-ecd2cea783cb}} \\
			{\footnotesize \texttt{1361402822.110498s 1 d1ffde43-f3d9-107b-aa2d-7e4e1ff88b46}}
%		\item Neat tool: \texttt{git annex sync}
%		\begin{itemize}
%			\item \texttt{git commit}
%			\item \texttt{git merge synced/master}
%			\item \texttt{git union-merge git-annex origin/git-annex refs/heads/git-annex}
%			\item \texttt{git push \$remote master:synced/master}
%		\end{itemize}
	\end{itemize}
\end{frame}

\begin{frame}
	\frametitle{Data integrity}
	\begin{itemize}
		\item SHA1, SHA2-\{224,256,384,512\} for integrity
		\item Set minimal number of required copies per suffix, directory, etc
		\item All remotes and special remotes can be verified
		\begin{itemize}
			\item remotes verify locally and transmit the result
			\item special remotes have to transfer all data to verify
		\end{itemize}
		\item Verification takes required amount of copies into account
		\item You can \textit{always} get your data out of a broken annex
		\begin{itemize}
			\item All data is stored as normal files on disk
			\item Symlinks work without git-annex
			\item git-annex objects carry their own checksum in their filename
		\end{itemize}
	\end{itemize}
\end{frame}

\begin{frame}
	\frametitle{Special remotes 1/2}
	\begin{itemize}
		\item Stores data in non-git-annex remotes
		\item Still tracks all data stored in special remotes
		\item Supports encryption for storage on untrusted machines/media
		\item Hook system lets you write to and read from arbitrary remotes
	\end{itemize}
\end{frame}

\begin{frame}
	\frametitle{Special remotes 2/2}
	\begin{itemize}
		\item Amazon Glacier
		\item Amazon S3
		\item bup
		\item directory
		\item rsync
		\item webdav
		\item web (media.ccc.de, Project Gutenberg, archive.org, etc)
		\item hook
		\begin{itemize}
			\item IMAP
			\item Tahoe-LAFS
		\end{itemize}
	\end{itemize}
\end{frame}

\begin{frame}
	\frametitle{git-annex assistant}
	\begin{itemize}
		\item Financed via \url{kickstarter.com}
		\item One year of dedicated programming by Joey Hess
		\item Daemon that adds data to the repository and syncs it between other repositories
		\item Web GUI on localhost
		\item Content notification via XMPP/Jabber
		\item Advanced ruleset for content distribution
		\item Direct mode without symlinks; especially useful for \\
			Mac OS X and Windows
		\item Android \& Windows ports
	\end{itemize}
\end{frame}


\subsection{Use cases}

\begin{frame}
	\frametitle{The Archivist}
	\begin{itemize}
		\item Put data into git-annex
		\item Distribute data among any number of drives, tapes, remotes, etc
		\item Store offline media in a safe place
		\item Maintain full information about number and location of all copies
	\end{itemize}
\end{frame}

\begin{frame}
	\frametitle{Media consumption}
	\begin{itemize}
		\item Import podcasts, videos, and slides
		\item Sync or export to consumption devices
		\item Consume media
		\item Drop consumed media from annex
		\item Deletion propagates through all annexes over time
	\end{itemize}
\end{frame}

\begin{frame}
	\frametitle{The Nomad}
	\begin{itemize}
		\item Keep copies of data on the Internet
		\item Optionally sync between several local devices for backup
		\item Add data locally and/or remotely while on the road
		\item Sync data between local and remote once at an Internet café or similar
		\item Perfect for photos while travelling
	\end{itemize}
\end{frame}

\begin{frame}
	\frametitle{Create different views or sets of the same data}
	\begin{itemize}
		\item Sometimes, you disagree with other people about the best way to organize data
		\item Different repositories can show a different view of the same data
			\begin{itemize}
				\item Completely delete some files or file types, for example RAW files
				\item Rename files and directories
			\end{itemize}
		\item Maintain a rebasing branch on top of the remote: \\
			\texttt{git config branch.master.rebase true}
	\end{itemize}
\end{frame}


\section{bup}

\begin{frame}
	\frametitle{Outline}
	\tableofcontents[currentsection]
\end{frame}

\begin{frame}
		\begin{center}
			\vfill
			\vfill
			\textbf{bup}
			\vfill
			\textit{Highly efficient file backup system based on the Git packfile format}
			\vfill
			\vfill
		\end{center}
\end{frame}

\subsection{One-slide-overview}

\begin{frame}
	\frametitle{In a word...}
	\begin{itemize}
		\item Written in Python
		\item Fast
		\item Very space-efficient (reduced 120 GiB (rsnapshot) to 45 GiB)
		\item Built-in de-duplication
		\item Can be mounted via FUSE
		\item Can not drop old data (there is a branch that supports this)
	\end{itemize}
\end{frame}


\section{Zsh}

\begin{frame}
	\frametitle{Outline}
	\tableofcontents[currentsection]
\end{frame}

\begin{frame}
		\begin{center}
			\vfill
			\vfill
			\textbf{Zsh}
			\vfill
			\textit{Best shell available. Period.}
			\vfill
			\vfill
		\end{center}
\end{frame}

\subsection{Why is this in a Git talk?}

\begin{frame}
	\frametitle{Not based on Git, but makes your life easier}
	\begin{itemize}
		\item Extremely powerful tab completion for the tools in this talk (and others!)
		\item Versatile left \emph{and right} prompts
		\item \texttt{vcs\_info}
		\begin{itemize}
			\item Displays information about the current VCS working copy in prompt
			\item Lots of customization options
			\item Supports Git, Bazaar, codeville, CVS, Darcs, fossil, GNU Arch, Mercurial, monotone, Perforce, Subversion, and svk
		\end{itemize}
		\item Can mimic Bash, Ksh, tcsh, etc.
		\item Too many other reasons to list (literally...)
	\end{itemize}
\end{frame}


\section{mr}

\begin{frame}
	\frametitle{Outline}
	\tableofcontents[currentsection]
\end{frame}

\begin{frame}
		\begin{center}
			\vfill
			\vfill
			\textbf{mr}
			\vfill
			\textit{Multiple Repository management tool}
			\vfill
			\vfill
		\end{center}
\end{frame}

\subsection{Too many repos...?}

\begin{frame}
	\frametitle{Tying it all together}
	\begin{itemize}
		\item Written in Perl
		\item Run bulk pull, push, and custom commands all, some, or one of your repositories
		\item Supports Git, vcsh, Bazaar, CVS, Darcs, fossil, git-svn, Mercurial, Subversion, unison, and veracity
		\item Trivial to extend to support more VCSs
		\item If you want to try all this, why not \texttt{vcsh clone} my mr repository template and run \texttt{mr up} to pull my Zsh config via vcsh?
	\end{itemize}
\end{frame}

\begin{frame}
	\frametitle{Suggested mr layout}
	\texttt{ \\
		\% cat \~{}/.mrconfig \\
		include = cat \~{}/.config/mr/config.d/* \\
		\% ls .config/mr/available.d \\
		mr.vcsh \\
		zsh.vcsh \\
		... \\
		\% ls -l .config/mr/config.d \\
		mr.vcsh -> ../available.d/mr.vcsh \\
		zsh.vcsh -> ../available.d/zsh.vcsh \\
		... \\
		\%
	}
\end{frame}


\section{Outro}

\begin{frame}
	\frametitle{Outline}
	\tableofcontents[currentsection]
\end{frame}

\subsection{Wrapping up}

\begin{frame}
	\frametitle{The final pitch...}
	\vfill
	I need literally less than five minutes of Internet access to sync my entire digital life while on the road.
	\vfill
\end{frame}

\subsection{Further reading}

\begin{frame}
	\frametitle{Project websites}
	Most of these are packaged for the major distributions
	\begin{itemize}
		\item ikiwiki: \url{http://ikiwiki.info/}
		\item etckeeper: \url{http://joey.kitenet.net/code/etckeeper/}
		\item vcsh: \url{https://github.com/RichiH/vcsh}
		\item git-annex: \url{http://git-annex.branchable.com/}
		\item bup: \url{https://github.com/bup/bup}
		\item mr: \url{http://kitenet.net/~joey/code/mr/}
		\item Wiki around this topic: \url{http://vcs-home.branchable.com/}
	\end{itemize}
\end{frame}

\begin{frame}
	\frametitle{Previous talks}
	Previous talks, a bit more in-depth than this one and available as video download:
	\begin{itemize}
		\item vcsh: \\
			\url{http://fosdem.org/2012/schedule/event/vcsh}
		\item git-annex: \\
			\url{http://fosdem.org/2012/schedule/event/gitannex}
	\end{itemize}
\end{frame}

\subsection{The End!}

\begin{frame}
	\frametitle{Thanks!}
		\begin{center}
			\vfill
			Thank you for listening!\\
			\vfill
			Questions? Ask now or during dinner, both is fine.
			\vfill
			See slide footer for further contact information.\\
			\vfill
			\#vcs-home @ irc.oftc.net\\
			vcs-home@lists.madduck.net
			\vfill
		\end{center}
\end{frame}


\end{document}


%\begin{frame}
%	\frametitle{}
%	\begin{itemize}
%		\item 
%		\item 
%		\item 
%		\item 
%		\item 
%	\end{itemize}
%\end{frame}
