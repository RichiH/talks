\documentclass[t]{beamer}
\usepackage{helvet}
\usepackage{calc}
\usepackage[utf8]{inputenc}
\usepackage[english]{babel}

\usetheme{Ilmenau}

\setbeamercovered{transparent}
\setbeamertemplate{navigation symbols}{}

\usepackage{units}
\usepackage{amsbsy}
\usepackage{amsmath}
\usepackage{amssymb}
\usepackage{graphics}
\usepackage{graphicx}
\usepackage{epsf}
\usepackage{epsfig}
\usepackage{fixmath}
%\usepackage{pgfmath}
\usepackage{wrapfig}


\title{From Zero to OpenStack}
\author{Richard Hartmann,\\
RichiH@\{freenode,OFTC,IRCnet\},\\
rh@globalways.net}
\institute{Globalways AG}
\date{August 20, 2011}



\begin{document}


\begin{frame}
	\titlepage
\end{frame}

\begin{frame}
	\frametitle{Outline}
	\tableofcontents
\end{frame}


\section{Quick intro}

\begin{frame}
	\frametitle{Outline}
	\tableofcontents[currentsection]
\end{frame}

\begin{frame}
	\frametitle{Who am I?}
	\begin{itemize}
		\item Project \& Network Operations Manager at Globalways AG
		\item Passionate about FLOSS
		\item freenode staff
	\end{itemize}
\end{frame}

\begin{frame}
	\frametitle{What is Globalways?}
	\begin{itemize}
		\item ISP \& ICTP
		\item German-wide DSL network
		\item Own DC in Stuttgart
		\item Total of six POPs in Frankfurt, Stuttgart \& Reutlingen
		\item Connected via redundant dark fibers, wavelengths \& leased lines
		\item Several high-profile government and industry clients
		\item Almost all services virtualized with Debian on Xen
		\item Always looking for new people...
	\end{itemize}
\end{frame}


\section{Current situation}

\begin{frame}
	\frametitle{What is the cloud?}
	\begin{itemize}
		\item Common buzz-word
		\item On-demand services
		\item Dynamic and elastic; assumption of unlimited resources
		\item Pay-as-you-go
		\item Different layers of magic
		\begin{itemize}
			\item Physical infrastructure
			\item Software
			\item Business processes
			\item End-user interaction
		\end{itemize}
		\item Everything-as-a-Service
	\end{itemize}
\end{frame}

\begin{frame}
	\frametitle{Problems with existing solutions}
	\begin{itemize}
		\item Expensive once you reach certain thresholds of usage
		\item Vendor lock-in
		\item You can not deploy a cloud solution used by a third-party, yourself
		\item Thus, no possibility to mix your own cloud with identical third-party cloud seamlessly
		\item No private cloud unless you are a huge customer (government, etc)
	\end{itemize}
\end{frame}

\begin{frame}
	\frametitle{Problems with existing solutions, contd.}
	\begin{itemize}
		\item No true QoS in the back-bone
		\item No guarantee for minimum IOPS
		\item No on-site, in-person verification of compliance (locked cages, access control, etc)
		\item You may be breaking European/German simply by using any cloud service provided by a US company or any of its subsidiaries, without being aware of it... (PATRIOT Act)
	\end{itemize}
\end{frame}


\section{What is OpenStack?}

\begin{frame}
	\frametitle{How does OpenStack solve this?}
	\begin{itemize}
		\item It's FLOSS
		\item Open collaboration, everyone is welcome
		\item Your data stays yours; just export here and import there
		\item You can create your own cloud, use a third-party, or a mix thereof
		\item Picking from a wide array of service providers allows you to choose based on the features you need, e.g. compliance, QoS, etc.
		\item European providers are not subject to the PATRIOT Act!
	\end{itemize}
\end{frame}

\begin{frame}
	\frametitle{How does OpenStack solve this?}
	\begin{itemize}
		\item Modular design; pick and mix the OpenStack projects you want \& need
		\item Heavy use of sharding
		\item Highly asynchronous design
		\item Eventual consistency is considered acceptable
		\item Emphasis on reliability and error-avoidance
		\item Blueprints, unit tests \& continuous integration
	\end{itemize}
\end{frame}

\begin{frame}
	\frametitle{Design goals of all OpenStack projects}
	\begin{itemize}
		\item Highly modularized: Add new functionality easily and quickly
		\item High availability: Scale to high workloads without failing
		\item Fault tolerance: Isolate faults automatically to minimize effects and cascades
		\item Recoverable: Diagnose, debug, \& rectify faults quickly
		\item Transparent \& public governance
		\item Open standards: Community-driven, RESTful API
	\end{itemize}
\end{frame}

\begin{frame}
	\frametitle{History of OpenStack}
	\begin{itemize}
		\item 2010-03: Rackspace decides to open source Rackspace Cloud
		\item 2010-05: NASA opens Nebula to the public
		\item 2010-07: Formal launch of the project \& first design summit in Austin
		\item 2010-10: Release of Austin
		\item 2011-02: Release of Bexar
		\item 2011-04: Release of Cactus
		\item 2011-08: Over 100 developers working on OpenStack
		\item Young, aggressive, efficient project
	\end{itemize}
\end{frame}

\begin{frame}
	\frametitle{Projects within OpenStack}
	\begin{itemize}
		\item Core
		\begin{itemize}
			\item Glance: Manage virtual machine images
			\item Nova: Manage virtualization solutions, e.g. Xen, KVM, UML, LXC, VMWare...
			\item Swift: Object storage
		\end{itemize}
		\item Incubation
		\begin{itemize}
			\item Keystone: Authentication frame-work
			\item Dashboard: web UI
		\end{itemize}
		\item More to come...
	\end{itemize}
\end{frame}


\section{OpenStack: Swift}

\begin{frame}
	\frametitle{Swift}
	\begin{itemize}
		\item Object storage
		\item No inherent limit on object sizes
		\item Supports striped transfers to increase speed
		\item Built-in redundancy
		\item Compability layer to emulate Amazon S3
		\item Scales extremely well
	\end{itemize}
\end{frame}

\begin{frame}
	\frametitle{Layout of Swift components}
	\begin{itemize}
		\item Auth nodes (obvious...)
		\item Proxy nodes
		\begin{itemize}
			\item Central gateway for access control, etc
			\item Handles most faults within Swift
			\item Does not cache data, forwards only
			\item Optional rate-limiting capability
		\end{itemize}
		\item Storage nodes
		\begin{itemize}
			\item Store accounts, containers and objects
			\item Main part of Swift
		\end{itemize}
		\item Ring
		\begin{itemize}
			\item Keeps everything together
			\item Distributed hash; determines storage locations
			\item Partitioned to keep memory foot-print small
		\end{itemize}
	\end{itemize}
\end{frame}

\begin{frame}
	\frametitle{Layout of Swift components}
	\begin{itemize}
		\item Replicators
		\begin{itemize}
			\item DB replicator
			\begin{itemize}
				\item Hash-based synchronization
				\item Resynchronisation via trivial high-water mark
				\item Simply looks at DB uid, DB-local id, timestamp and hash
				\item Initial seed to new machines via plain rsync!
			\end{itemize}
			\item Storage replicator
			\begin{itemize}
				\item Hash-based, as well
				\item Synchronization based on rsync
				\item Use of partitioned rings keeps directory indices small and thus rsync fast
			\end{itemize}
		\end{itemize}
	\end{itemize}
\end{frame}

\begin{frame}
	\frametitle{Layout of Swift components}
	\begin{itemize}
		\item Account Reaper
		\begin{itemize}
			\item Asynchronous
			\item Allows for lazy deletion!
		\end{itemize}
		\item Auditor
		\begin{itemize}
			\item Asynchronous
			\item Trawls the complete storage network
			\item Quarantines and replaces bad objects automagically
			\item Absolute must; failed disks are easy, bit rot is hard!
		\end{itemize}
	\end{itemize}
\end{frame}

\begin{frame}
	\frametitle{Initial deployment}
	\begin{itemize}
		\item Simply install Ubuntu LTS 10.04 and add the OpenStack repository
		\item Debian packages coming along, as well
		\item Dell's Crowbar is looking very promising to automate most of deployment; based on Opscode's Chef Server
		\item URLs with detailed instructions at the end of this presentation :)
	\end{itemize}
\end{frame}

\begin{frame}
	\frametitle{Deployment best practices}
	\begin{itemize}
		\item Five or more fully separate zones
			\begin{itemize}
				\item Physical location
				\item Power and backup power
				\item Network connectivity
				\item WAN interconnects
				\item You can start with one machine per zone; with proper planning, you can simply add machines to the zones to scale as you grow
			\end{itemize}
		\item Three or more copies of every object, at max one per zone
		\item Even if you lose a whole zone, you can replicate from two zones to two others, essentially cutting replication time in half
	\end{itemize}
\end{frame}

\begin{frame}
	\frametitle{Deployment best practices, contd.}
	\begin{itemize}
		\item Separate management network from data network
		\item VLANs are OK, two physical interfaces preferable
		\item Never use RAID; failures should be known to Swift ASAP!
		\item Contain maintenance, upgrades etc to one zone at a time
		\item Quick replacement/repair? Just do it. Might take longer? Take disk/machine out of order so Swift can start to replicate in the background
		\item Swift provides good fault detection \& handling; still monitor on top of that, yourself!
		\begin{itemize}
			\item OSSEC to aggregate \& parse log files
			\item Zabbix, Nagios, Ganglia, etc. for everything you can think of (e.g. load, smart status, sensors, disk space, port status...)
		\end{itemize}
	\end{itemize}
\end{frame}

\begin{frame}
	\frametitle{Our deployment}
	\begin{itemize}
		\item Still in the testing phase, but looking very good
		\item Five zones across three data centers
		\item 1 Gbit/s switching within zones
		\item Zones interconnected via MPLS over redundant 10 Gbit/s lines
		\item Switching \& MPLS ensure line-rate speeds with minimal overhead
	\end{itemize}
\end{frame}


\section{Conclusion}

\begin{frame}
	\frametitle{Is it worth it?}
	\begin{itemize}
		\item Short answer: Yes
		\item Yet, OpenStack is still somewhat intimidating
		\item The next release, Diablo, is scheduled for September and considered ready for general consumption
		\item Working with a well-designed product is fun
		\item You know the people who build and use OpenStack care about your data
	\end{itemize}
\end{frame}

\begin{frame}
	\frametitle{More resources}
	\begin{itemize}
		\item \url{http://www.openstack.org}
		\item \url{http://wiki.openstack.org}
		\item \url{http://glance.openstack.org}
		\item \url{http://nova.openstack.org}
		\item \url{http://swift.openstack.org}
		\item \url{http://tinyurl.com/openstack-releases}
		\item \url{http://tinyurl.com/openstack-admin-cactus}
		\item \url{https://github.com/dellcloudedge/crowbar}
		\item \url{http://www.opscode.com}
	\end{itemize}
\end{frame}

\begin{frame}
	\frametitle{Thanks!}
	\begin{itemize}
		\item I hope you had fun listening to this talk
		\item Feel free to contact me with questions
	\end{itemize}
\end{frame}



\end{document}
