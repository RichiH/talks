\documentclass[t]{beamer}
\usepackage{helvet}
\usepackage{calc}
\usepackage[utf8]{inputenc}
\usepackage[english]{babel}

\usetheme{Ilmenau}

\setbeamercovered{transparent}
\setbeamertemplate{navigation symbols}{}

\usepackage{units}
\usepackage{amsbsy}
\usepackage{amsmath}
\usepackage{amssymb}
\usepackage{graphics}
\usepackage{graphicx}
\usepackage{epsf}
\usepackage{epsfig}
\usepackage{fixmath}
%\usepackage{pgfmath}
\usepackage{wrapfig}


\title{Monitoring with Prometheus \& Grafana}
\subtitle{One year of Prometheus}
\author{Richard Hartmann,\\
RichiH@\{freenode,OFTC,IRCnet\},\\
richih@\{fosdem,debian,richih\}.org,\\
richard.hartmann@space.net}
\date{2016-11-22}


\begin{document}

% hide all subsections
\setcounter{tocdepth}{1}

\begin{frame}
	\titlepage
\end{frame}

%\begin{frame}
%	\frametitle{Statistics}
%	\tableofcontents
%\end{frame}

\section{Introduction}

\subsection{}

\begin{frame}
	\frametitle{`whoami`}
	\begin{itemize}
		\item Richard "RichiH" Hartmann
		\item System architect at SpaceNet AG
		\item FOSDEM, DebConf, DENOGx, PromCon staff
		\item Debian Developer
		\item Author of \url{https://github.com/RichiH/vcsh}
		\item Always looking for nice co-workers in the Munich area
	\end{itemize}
\end{frame}


\section{Background}

\subsection{}

% TODO kill this?
\begin{frame}
	\frametitle{`whowasi`}
	\begin{itemize}
		\item 2009 - 2015: Solely responsible for a Germany-wide backbone's
		\begin{itemize}
			\item Architecture
			\item Purchasing
			\item Operations
			\item ...and On-Call for 24 * 365 * 7 hours
		\end{itemize}
		\item Literally, my sanity depended on aggressive, yet correct, monitoring \& alerting
		\item Love monitoring, but despise (almost) all monitoring tools
		\item Used Zabbix exclusively
	\end{itemize}
\end{frame}

% TODO Change?
\begin{frame}
	\frametitle{SpaceNet}
	\begin{itemize}
		\item SpaceNet is the oldest commercial ISP in Germany; operating since 1993
		\item Legacy, in-house solutions which predate everything else
		\item One company-wide monitoring solution: watchdog \& watchcat
		\item Powerful and efficient, but alerting done through B52-style email carpet bombing
		\item Every team has its own custom tools on top
		\item Islands of data: no APIs, no machine-readable export
	\end{itemize}
\end{frame}

\subsection{}

% TODO
%		\item Delta-of-delta data storage
% instrumentation, metrics collection (flaoat64 only), querying, alerting, dashboarding, graphing, trending
%   storage efficiency

% one exporter per service
% focus on services
% failures happen at scale
% automates away toil

% SQL vs PromQL von jrv? - vector math wie R
% HA? run two
% federation
% kein auth/encryption -> reverse proxy
%PromQL: avg by(city) (temperature_celsius{country=”germany”})
%SQL: SELECT city, AVG(value) FROM temperature_celsius WHERE country=”germany” GROUP BY city
%PromQL: rate(errors{job=”foo”}[5m]) / rate(total{job=”foo”}[5m])
%SQL:
%SELECT errors.job, errors.instance, […more labels…], rate(errors.value, 5m) / rate(total.value, 5m) FROM errors JOIN total ON […all the label equalities…] WHERE errors.job=”foo” AND total.job=”foo”

\begin{frame}
	\frametitle{Seeing the light}
	\begin{itemize}
		\item Ran DebConf15 on LibreNMS, wanted to do the same for SpaceNet \& FOSDEM 2016
		\item 2015-10-01: Inform FOSDEM team of planned migration
		\item 2015-10-02: Murali Suriar suggests Prometheus instead
		\item 2015-10-03: PoC at SpaceNet and submit first patch
		\item 2016-01-29: Hackday to migrate FOSDEM
	\end{itemize}
\end{frame}


\begin{frame}
	\frametitle{Prometheus 101}
	\begin{itemize}
		\item Inspired by Google's Borgmon
		\item Time series database
		\item float64 timestamp, float64 value
		\item Instrumentation \& exporters
		\item No event logging
	\end{itemize}
\end{frame}

\begin{frame}
	\frametitle{Main selling points}
	\begin{itemize}
		\item Highly dynamic, built-in service discovery
		\item No hierarchical model, n-dimensional label set
		\item PromQL: for processing, graphing, alerting, and export
		\item Simple operation
		\item Highly efficient
	\end{itemize}
\end{frame}

\begin{frame}
	\frametitle{Efficiency}
	\begin{itemize}
		\item 64 GiB RAM, 32 cores, 525,000 samples/second
		\item 0.066 bit/sample real world best case
		\item 1.28 byte/sample average at SoundCloud
		\item Cheap ingestion \& storage means more data for you
	\end{itemize}
\end{frame}

\begin{frame}[fragile]
	\frametitle{Exposition format}
	\fontsize{10pt}{12}\selectfont
	\begin{verbatim}
http_requests_total{env="prod",method="post",code="200"} 1027
http_requests_total{env="prod",method="post",code="400"} 3
http_requests_total{env="prod",method="post",code="500"} 12
http_requests_total{env="prod",method="get",code="200"} 20
http_requests_total{env="test",method="post",code="200"} 372
http_requests_total{env="test",method="post",code="400"} 75
	\end{verbatim}
\end{frame}

\begin{frame}[fragile]
	\frametitle{PromQL}
	\begin{verbatim}
avg by(city) (temperature_celsius{country=”germany”})

SELECT city, AVG(value) FROM temperature_celsius WHERE \
 country=”germany” GROUP BY city

rate(errors{job=”foo”}[5m]) / rate(total{job=”foo”}[5m])

SELECT errors.job, errors.instance, […more labels…], \
 rate(errors.value, 5m) / rate(total.value, 5m) FROM \
 errors JOIN total ON […all label equalities…] WHERE \
 errors.job=”foo” AND total.job=”foo”
	\end{verbatim}
\end{frame}


\section{Lightbringer}

\subsection{Technical challenges}

\begin{frame}
	\frametitle{Here, there be networks}
	\begin{itemize}
		\item We are an ISP, remember?
		\item Roughly 1000 devices polled via SNMP
		\item Might be the world's largest snmp\_exporter installation
		\item Python implementation at pathologic system load
		\begin{itemize}
			\item It goes up to eleven...
			\item About 60/300 devices flapping
			\item Set of affected devices stable
			\item Never found root cause
		\end{itemize}
	\end{itemize}
\end{frame}

\begin{frame}
	\frametitle{Solution}
	\begin{itemize}
		\item Contracted Brian Brazil to reimplement in Go
		\item Go implementation hit some unexpected pitfalls of real life SNMP
		\begin{itemize}
			\item Some data structures returned repeatedly
			\item Duplicate identifiers
			\item Table indices emtpy
		\end{itemize}
		\item Go errors out completely for those
		\item Still using Python for affected devices
	\end{itemize}
\end{frame}

\begin{frame}
	\frametitle{Caveats}
	\begin{itemize}
		\item InetAddress broken in Python
		\item IOS XR non-standard layout not fully supported yet 
		\item Some devices die when polled too often
		\vfill
			That was easy; let's have lunch!
	\end{itemize}
\end{frame}


\subsection{Social Challenges}

\begin{frame}
	\frametitle{The biggest challenge}
		\begin{center}
			\vfill
			The hardest problems to solve are the social ones.
			\vfill
		\end{center}
\end{frame}

\begin{frame}
	\frametitle{Resistance to change}
	\begin{itemize}
		\item Incentives often run counter to change
		\item Change is hard
		\item Unless processes embrace and automate change
		\item Trade-off between delayed/disputed payoff during transition
		\item Due diligence: Critical systems run in parallel for some time
	\end{itemize}
\end{frame}

\begin{frame}
	\frametitle{Toil}
	\begin{itemize}
		\item If teams are busy firefighting, they don't have time to engineer
%		\item Keep legacy working, but have clean path forward
		\item Keep extra effort on the team low, if possible
		\item Strive for immediate benefits
		\item Focus on removing repeated, manual tasks of no lasting benefit
		\item Show that you free up time and reduce toil
% operations stakeholders fear change
  % they get paid to keep the current system up and running, not to increase change
% everyone wants to live in a SRE world with error budgets and caps on toil
%Toil in too many places
	\end{itemize}
\end{frame}

\begin{frame}
	\frametitle{Sanity \& sleep}
	\begin{itemize}
		\item If it's not actionable, it's not an alert
		\item If it's not urgent, it's not an alert
		\item Important, but not urgent, stuff is handled during business hours
		\item Predict your usage so you add capacity during business hours
		\item If there's no playbook, it does not go into production
		\item If a service does not have proper SLOs and alerts, it does not go into production
	\end{itemize}
\end{frame}

\begin{frame}
	\frametitle{That one mailserver incident...}
	\begin{itemize}
		\item Wrong flag in config
		\item One server accepting outside mail
		\item Spammers do a clean, staggered ramp-up
		\item Once they go all-in the mail gateways come under heavy load
		\item Quote from On-Call "It took me less than 30 seconds to figure out the problem; with our old system it would have taken at least 60 minutes"
		\item ...and all of a sudden, you have buy-in from a few more people
	\end{itemize}
\end{frame}

\begin{frame}
	\frametitle{Perspective \& Incentives}
	"An engineer can talk for hours about a function; try that with the CEO"
	\vfill
	\begin{itemize}
		\item Managers: revenue, process execution
		\item Architects: clean design, process defintion
		\item Product/Service owners: Powerful dashboards
		\item Team leads: morale, quick execution
		\item Operators: reduce toil, increase sleep
		\item Tell everyone what they need to hear (but never lie)
%  Different parts of the picture for different people and languages
%  Anyone can talk to an engineer about a UPS for two hours, but management...
	\end{itemize}
	\vfill
\end{frame}

\begin{frame}
	\frametitle{Big Picture}
	\begin{itemize}
		\item Put a big picture on the (proverbial) wall
		\item Show everyone the pieces they care about
		\item Make sure to play to their intrinsic motivation
		\item Get buy-in
		\item Going forward, align steps with that picture
		\item Distributed alignment with goals across teams
	\end{itemize}
\end{frame}
% money is made with brownfield so you need to make sure there is no impact
  % lots of overlap until new stuff gains trust
  % people's motivation is to resist change; and that's OK until the system can actually handle change
  % most will readily agree that the new approach is better, but they fear the way there
    % ideally, we should magic from zero to done without anyone noticing

\begin{frame}
	\frametitle{Leverage}
	\begin{itemize}
		\item One combined system allows for correlation and combination
		\item Power usage against service load
		\item Optical networks against outside temperature
		\item Datacenter power feed load against new deployments
		\item ...and lots more
	\end{itemize}
\end{frame}

\begin{frame}
	\frametitle{Leverage}
	\begin{itemize}
		\item One source of truth for
		\begin{itemize}
			\item Tactical overview for current state
			\item Dashboards for drill-down
			\item Auto-generated PDFs for customers
			\item Global SLO statements for sales
			\item Usage exports for accounting
		\end{itemize}
	\item If all you have is a hammer... choose your hammer well
%Give them what they want/play to their intrinsic motivation
%%%%% dopplung zu persepcitve!
% sales
% accounting
% power count
%%%%%
	\end{itemize}
\end{frame}

%\begin{frame}
%	\frametitle{Caveat}
%	\begin{itemize}
%		\item In brownfield, revenue comes from legacy systems
%		\item If you impact revenue, all buy-in is gone
%	\end{itemize}
%\end{frame}


\section{Outro}

\subsection{Outlook}

\begin{frame}
	\frametitle{TODO}
	\begin{itemize}
		\item Merge config management across teams
		\item Adapt machines and services to modern orchestration
		\begin{itemize}
			\item Highly fractured and specific customer setups
			\item Revenue comes from those brownfield installation
			\item Finding the correct balance will be tricky
		\end{itemize}
		\item Adopt error budgets
		\item Hire more people. Munich is beautiful!
	\end{itemize}
\end{frame}



%If no motivation, escalate



%Know of UPS / core link downtime within less than 30 seconds, but no false positives
%Want USV in seconds, Mail Accounting reicht täglich

%Goals / that's obvious section?

%Current state NOW, not five minutes ago (but info)
%Important later, during daytime
%Make sure tickets are opened so we can postmortem all the things

%Push vs pull not a conceptual problem due to watchdog
%  Concept of scraping new
%Watchdog can do logic checks, regexp etc; not only metrics
%

%Wollen ewiges Storage
%Immer noch zwei große Prom, Teams noch nicht so weit.

%Too much "fix this now" and not learning aggressively from outages.
%Not all teams tracked and compared KPIs over time
%From plain uptime to service request success rate

%Dashboard
%Perpetual greenfield
%One tool fits all
%Make your hammer fit your nails


\subsection{}

\begin{frame}
	\frametitle{Thanks!}
		\begin{center}
			\vfill
			Thanks for listening!\\
			\vfill
			Questions?
			\vfill
			See slide footer for contact information.
			\vfill
		\end{center}
\end{frame}

\end{document}

%\begin{frame}
%	\frametitle{}
%	\begin{itemize}
%		\item 
%		\item 
%		\item 
%		\item 
%		\item 
%	\end{itemize}
%\end{frame}
