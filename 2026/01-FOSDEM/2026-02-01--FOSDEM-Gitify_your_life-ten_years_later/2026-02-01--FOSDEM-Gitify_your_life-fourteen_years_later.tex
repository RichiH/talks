\documentclass[t]{beamer}
\usepackage{helvet}
\usepackage{calc}
\usepackage[utf8]{inputenc}
\usepackage[english]{babel}

\usetheme{Ilmenau}

\setbeamercovered{transparent}
\setbeamertemplate{navigation symbols}{}

\usepackage{units}
\usepackage{amsbsy}
\usepackage{amsmath}
\usepackage{amssymb}
\usepackage{graphics}
\usepackage{graphicx}
\usepackage{epsf}
\usepackage{epsfig}
\usepackage{fixmath}
%\usepackage{pgfmath}
\usepackage{wrapfig}


\title{Gitify your life - 14 years later}
\subtitle{web, blog, configs, data, and backups}
\author{Richard Hartmann,\\
@RichiH@chaos.social}
\date{2026-02-01}



\begin{document}

% hide all subsections
\setcounter{tocdepth}{1}

\begin{frame}
	\titlepage
\end{frame}

%\begin{frame}
%	\frametitle{Outline}
%	\tableofcontents
%\end{frame}


\section{Intro}

%\subsection{Personal stuff}
%
%\begin{frame}
%	\frametitle{Who am I?}
%	\begin{itemize}
%		\item Richard "RichiH" Hartmann
%		\item Backbone and project manager at Globalways AG
%		\item freenode \& OFTC staff
%		\item Debian Developer
%		\item Author of vcsh \& metamonger
%%		\item Passionate about FLOSS
%	\end{itemize}
%\end{frame}

\subsection{The basics}

\begin{frame}
	\frametitle{What is Git?}
	\begin{itemize}
		\item Version control system
		\item Distributed
		\begin{itemize}
			\item No need for central repository
			\item Allows you to commit while offline
		\end{itemize}
		\item Stores commits (parent commit reference, commit message, root tree object) and tree objects (blobs and other tree objects)
		\item Light-weight branches
		\item pre-/post-action hooks
		\item Full history in every checkout
	\end{itemize}
\end{frame}


\section{etckeeper}

\begin{frame}
		\begin{center}
			\vfill
			\vfill
			\textbf{etckeeper}
			\vfill
			\textit{etckeeper is a collection of tools to let /etc be stored in a Git, Mercurial, Darcs, or Bazaar repository}
			\vfill
			\textit{\url{https://etckeeper.branchable.com}}
			\vfill
			\vfill
		\end{center}
\end{frame}

\subsection{Minimal, quick overview}

\begin{frame}
	\frametitle{In a word}
	\begin{itemize}
		\item Implemented in POSIX shell
		\item Auto-commits /etc prior to and after all actions by package manager
		\item Hooks into apt, yum, pacman-g2, and cron
		\item Allows manual commits
		\item Various back-ends
		\begin{itemize}
			\item Bazaar
			\item Darcs
			\item Git
			\item Mercurial
		\end{itemize}
		\item Easy way to recover from failures, misconfiguration or to clone machines
	\end{itemize}
\end{frame}


\section{bup}

\begin{frame}
		\begin{center}
			\vfill
			\vfill
			\textbf{bup}
			\vfill
			\textit{Highly efficient file backup system based on the Git packfile format}
			\vfill
			\textit{\url{https://bup.github.io}}
			\vfill
			\vfill
		\end{center}
\end{frame}


\section{ikiwiki}

\begin{frame}
		\begin{center}
			\vfill
			\vfill
			\textbf{ikiwiki}
			\vfill
			\textit{ikiwiki is a wiki compiler. It converts wiki pages into HTML pages suitable for publishing on a website}
			\vfill
			\textit{\url{https://ikiwiki.info}}
			\vfill
			\vfill
		\end{center}
\end{frame}

\begin{frame}
		\begin{center}
			\vfill
			\vfill
			\textbf{Hugo}
			\vfill
			\textit{Currently don't have a personal website, but if I did, it would be in Git still}
			\vfill
			\textit{\url{https://gohugo.io}}
			\vfill
			\vfill
		\end{center}
\end{frame}


\section{git-annex}

\begin{frame}
		\begin{center}
			\vfill
			\vfill
			\textbf{git-annex}
			\vfill
			\textit{manage files with Git, without checking their contents in}
			\vfill
			\textit{\url{https://git-annex.branchable.com}}
			\vfill
			\vfill
		\end{center}
\end{frame}

\subsection{Background}

\begin{frame}
	\frametitle{What is git-annex?}
	\begin{itemize}
		\item Based on Git
		\item Maintains metadata in Git, actual files in the annex
		\item Still allows you to check files into Git if you want to
		\item Written with low bandwidth and flaky connections in mind
		\item Superb data integrity, tracking, verification, and copying
		\item Supports special remotes such as HTTP, video hosting sites, S3, etc
	\end{itemize}
\end{frame}


\section{metamonger}

\begin{frame}
		\begin{center}
			\vfill
			\vfill
			\textbf{metamonger}
			\vfill
			\textit{Like metastore, but done right}
			\vfill
			\textit{\url{https://github.com/RichiH/metamonger}}
			\vfill
			\vfill
		\end{center}
\end{frame}


\section{vcsh}

\begin{frame}
		\begin{center}
			\vfill
			\vfill
			\textbf{vcsh}
			\vfill
			\textit{Version Control System for \$HOME \\ (multiple Git repositories in \$HOME)}
			\vfill
			\textit{\url{https://github.com/RichiH/metamonger}}
			\vfill
			\vfill
		\end{center}
\end{frame}

\subsection{Technical details}

\begin{frame}
	\frametitle{What is vcsh?}
	\begin{itemize}
		\item Implemented in POSIX shell
		\item Based on Git, but...
		\begin{itemize}
			\item Git is unable to maintain several working copies in one directory
			\item This is a safety feature...
			\item ...which sucks if you want to keep your configs in Git
		\end{itemize}
		\item vcsh uses fake bare Git repositories to work around this limitation
		\item Think of it as an extension to, or a wrapper around, Git
		\item Powerful and extensible hook system
		\item \url{https://github.com/RichiH/vcsh}
	\end{itemize}
\end{frame}


\section{mr}

\begin{frame}
		\begin{center}
			\vfill
			\vfill
			\textbf{mr (now \textit{myrepos})}
			\vfill
			\textit{a tool to manage all your version control repos}
			\vfill
			\textit{\url{https://myrepos.branchable.com}}
			\vfill
			\vfill
		\end{center}
\end{frame}

\subsection{Too many repos...?}

\begin{frame}
	\frametitle{Tying it all together}
	\begin{itemize}
		\item Run bulk pull, push, and custom commands all, some, or one of your repositories
		\item Supports Git, vcsh, Bazaar, CVS, Darcs, fossil, git-svn, Mercurial, Subversion, unison, and veracity
		\item Trivial to extend to support more VCSs
		\item If you want to try all this, why not \texttt{vcsh clone} my mr repository template and run \texttt{mr up} to pull my Zsh config via vcsh?
	\end{itemize}
\end{frame}


\section{Zsh}

\begin{frame}
		\begin{center}
			\vfill
			\vfill
			\textbf{Zsh}
			\vfill
			\textit{Best shell available. Period.}
			\vfill
			\textit{\url{https://www.zsh.org}}
			\vfill
			\vfill
		\end{center}
\end{frame}

\subsection{Why is this in a Git talk?}

\begin{frame}
	\frametitle{Not based on Git, but makes your life easier}
	\begin{itemize}
		\item Extremely powerful tab completion for the tools in this talk (and others!)
		\item Versatile left \emph{and right} prompts
		\item \texttt{vcs\_info}
		\begin{itemize}
			\item Displays information about the current VCS working copy in prompt
			\item Lots of customization options
			\item Supports Git, vcsh, Bazaar, codeville, CVS, Darcs, fossil, GNU Arch, Mercurial, monotone, Perforce, Subversion, and svk
		\end{itemize}
		\item Can mimic Bash, Ksh, tcsh, etc.
		\item Too many other reasons to list (literally...)
	\end{itemize}
\end{frame}



\subsection{The End!}

\begin{frame}
	\frametitle{Thanks!}
		\begin{center}
			\vfill
			Thank you for listening!\\
			\vfill
			Questions? Ask now or catch me after this talk, both is fine.
			\vfill
			See slide footer for further contact information.\\
			\vfill
			\url{https://github.com/richih/talks}
			\vfill
			@RichiH@chaos.social
			\vfill
		\end{center}
\end{frame}


\end{document}


%\begin{frame}
%	\frametitle{}
%	\begin{itemize}
%		\item 
%		\item 
%		\item 
%		\item 
%		\item 
%	\end{itemize}
%\end{frame}
