\documentclass[aspectratio=169]{beamer}
\usepackage{helvet}
\usepackage{calc}
\usepackage[utf8]{inputenc}
\usepackage[english]{babel}

\usetheme{Ilmenau}

\setbeamercovered{transparent}
\setbeamertemplate{navigation symbols}{}

\usepackage{units}
\usepackage{amsbsy}
\usepackage{amsmath}
\usepackage{amssymb}
\usepackage{graphics}
\usepackage{graphicx}
\usepackage{epsf}
\usepackage{epsfig}
\usepackage{fixmath}
%\usepackage{pgfmath}
\usepackage{wrapfig}


\title{On Observability}
\subtitle{}
\author{Richard Hartmann,\\
RichiH@\{freenode,OFTC,IRCnet\},\\
richih@\{debian,fosdem,richih\}.org,\\
@TwitchiH}
\date{2018-11-21}


\begin{document}

% hide all subsections
\setcounter{tocdepth}{1}

\section{Introduction}

\subsection{}

\begin{frame}
	\titlepage
\end{frame}

%\begin{frame}
%	\frametitle{Statistics}
%	\tableofcontents
%\end{frame}

\subsection{}

\begin{frame}
	\frametitle{`whoami`}
	\begin{itemize}
		\item Richard "RichiH" Hartmann
		\item Swiss army chainsaw at SpaceNet
		\begin{itemize}
			\item Leading the build of one of the most modern datacenters in Europe
			\item ...and always looking for nice co-workers in the Munich area
		\end{itemize}
		\item FOSDEM, DebConf, DENOGx, PromCon staff
		\item Author of \url{https://github.com/RichiH/vcsh}
		\item Debian Developer
		\item Prometheus team member
		\item OpenMetrics founder
	\end{itemize}
\end{frame}


\section{Intro}

\subsection{Definitions}

\begin{frame}
	\frametitle{Buzzword}
	\begin{center}
		\vfill
		\textit{buzzword}, n:\\
		A useful concept which has been picked up by everyone without understanding its deeper meaning and used so often that it's devoid of its original context and definition.\\
		May revert to usefulness in the same or different meaning, or die off.
		\vfill
	\end{center}
\end{frame}

\begin{frame}
	\frametitle{Cargo culting}
	\begin{center}
		\vfill
		\textit{cargo culting}, v:\\
		Villagers on remote Pacific islands observed U.S. soldiers building marker fires and runways during WWII; this made planes come and bring gifts from the heavens.
		Cults emerged which built bonfires and runways in the hopes of getting more gifts.\\
		Also see: \textit{copy \& paste}
		\vfill
	\end{center}
\end{frame}

\begin{frame}
	\frametitle{Monitoring}
	\begin{center}
		\vfill
		\textit{monitoring}, n:\\
		Old buzzword.\\
		Too often: focus is put on collecting, persisting, and alerting on just any data, as long as its data.\\
		It might also be garbage.\\
		Also see: \textit{data lake}
		\vfill
	\end{center}
\end{frame}

\begin{frame}
	\frametitle{Observability}
	\begin{center}
		\vfill
		\textit{observability}, n:\\
		Function of a system with which humans and machines can observe, understand, and act on the state of said system.
		\vfill
	\end{center}
\end{frame}

\begin{frame}
	\frametitle{Thanks!}
	\begin{center}
		\vfill
		Thanks for listening!\\
		\vfill
		Questions?
		\vfill
		Email me if you want a job in Munich.
		\vfill
		See slide footer for contact info.
		\vfill
	\end{center}
\end{frame}


\subsection{Outlook}

\begin{frame}
	\frametitle{Learnings}
	\vfill
	\begin{itemize}

		\item Baseline of monitoring
		\item Types of monitoring data and when to use them

		\item Types of complexity
		\item Containing complexity

		\item Service, contracts, SL\{I,O,A\}, etc
		\item Services upon services

		\item Bringing it all together
	\end{itemize}
	\vfill
\end{frame}



\section{Monitoring}


\subsection{Baseline of monitoring}

\begin{frame}
	\frametitle{Recap}
	\begin{center}
		\vfill
		Monitoring is the bedrock of everything (in IT).
		\vfill
		Hope is not a strategy.
		\vfill
	\end{center}
\end{frame}

\begin{frame}
	\frametitle{Claim}
	\begin{center}
		\vfill
		Uninformed, or cargo culted, monitoring equals hope.\\
		Also see: \textit{ISO 9001 \& 27001}
		\vfill
		So we need informed decisions, made on a factual basis.
		\vfill
	\end{center}
\end{frame}


\begin{frame}
	\frametitle{50:50}
	\begin{center}
		\vfill
		Broadly speaking, there are metrics and events
		\vfill
		Metrics: Changes over time
		\vfill
		Events: Specific points in time
		\vfill
	\end{center}
\end{frame}


\subsection{Metrics, events, and when to use them}

\begin{frame}
	\frametitle{Metrics}
	\begin{itemize}
		\item Numerical data
		\begin{itemize}
			\item Counters: Things going up monotonically, e.g. total transmitted bytes
			\item Gauges: Things going up and down, e.g. temperatures
			\item Bool/ENUM: Special case of gauges indicating a changing state or a singular event
			\item Histograms and percentiles: Things going into buckets or being in a specific percentage band, e.g. latency
		\end{itemize}
		\item Counters and histograms lose, or compress, data (in the common case)
		\item Easy to handle at scale
		\item You can do math on them!
	\end{itemize}
\end{frame}


\begin{frame}
	\frametitle{Logs}
	\begin{itemize}
		\item Most likely text items
		\item Usually with inlined metadata
		\item Scale linearly with service load
		\item Can be summarized into counters, histograms, and quantiles
	\end{itemize}
\end{frame}

\begin{frame}
	\frametitle{Traces}
	\begin{itemize}
		\item Execution path along the, hopefully annotated, code
		\item Impacts code runtime, aka expensive
		\item Can hide race conditions and other timing-dependent issues
		\item Usually disabled or sampled
	\end{itemize}
\end{frame}

\begin{frame}
	\frametitle{Dumps}
	\begin{itemize}
		\item Thrown when programs abort abnormally
		\item Execution path along the code
		\item Not annotated unless compiler artefacts of the exact same program are available
		\item You want to avoid them, but you also want to collect them when they happen
	\end{itemize}
\end{frame}

\begin{frame}
	\frametitle{When to use what}
	\begin{itemize}
		\item Metrics should usually be the first point of entry
		\begin{itemize}
			\item ..for alerts
			\item ..for dashboards
			\item ..for data exploration
		\end{itemize}
		\item Logs are usually the second step
		\begin{itemize}
			\item ..for establishing order of events
			\item ..for detailed information
			\item ..for access control, due diligence, etc
		\end{itemize}
		\item Traces and dumps are useful to understand why individual system components behave in a certain way
	\end{itemize}
\end{frame}



\section{Complexity}


\subsection{It may be rocket science}

\begin{frame}
	\frametitle{Types of complexity}
	\begin{center}
		\vfill
		Fake complexity, aka shitty design
		\vfill
		System-inherent complexity
		\vfill
	\end{center}
\end{frame}

\begin{frame}
	\frametitle{Handling complexity}
	\begin{center}
		\vfill
		You can reduce fake complexity
		\vfill
		You can contain inherent complexity
		\vfill
	\end{center}
\end{frame}

\begin{frame}
	\frametitle{Containing complexity}
	\begin{center}
		\vfill
		You need to compartmentalize complexity to make it manageable
		\vfill
	\end{center}
\end{frame}



\section{Services}


\subsection{Baseline of services}

\begin{frame}
	\frametitle{What's a service?}
	\begin{center}
		\vfill
		A service is anything a different entity relies upon
		\vfill
		This entity might be another team, a customer, or yourself
		\vfill
	\end{center}
\end{frame}

\begin{frame}
	\frametitle{Handover}
	\begin{center}
		\vfill
		Service delineations have many names: interface, API, contract
		\vfill
		I like to think of all of them as contracts. Why?
		\vfill
	\end{center}
\end{frame}


\subsection{Pop culture references}

\begin{frame}
	\frametitle{Tetris}
	\begin{center}
		\vfill
		Services build on top of each other
		\vfill
		(Network * x + machine/container/kubelet * y + daemon/microservice * z) * n = HTTP service
		\vfill
	\end{center}
\end{frame}

\begin{frame}
	\frametitle{Jenga}
	\begin{center}
		\vfill
		This tower can topple if the underlying building blocks are removed without due consideration.
		\vfill
		"Contract" implies a firm commitment, which is why I like this term.
		\vfill
	\end{center}
\end{frame}

\begin{frame}
	\frametitle{Chinese whispers}
	\begin{center}
		\vfill
		There's another common term for contract: layer.
		\vfill
		Imagine if someone simply changed how IP works.
		\vfill
	\end{center}
\end{frame}

\begin{frame}
	\frametitle{Trolling}
	\begin{center}
		\vfill
		For example, imagine someone would claim that IP addresses have 128 instead of 32 bits all of sudden...
		\vfill
	\end{center}
\end{frame}

\begin{frame}
	\frametitle{Cake}
	\begin{center}
		\vfill
		So we agree that layering makes sense, but why do we agree?
		\vfill
	\end{center}
\end{frame}

\begin{frame}
	\frametitle{It's complicated}
	\begin{center}
		\vfill
		Because we internalized that it's good practice to contain and compartmentalize system-inherent complexity.
		\vfill
	\end{center}
\end{frame}

\begin{frame}
	\frametitle{Spectre, Meltdown, etc}
	\begin{center}
		\vfill
		A CPU is highly complex, but we are happy to trust their hidden complexity because there's a well-defined service boundary.
		\vfill
	\end{center}
\end{frame}



\section{Observability}


\subsection{Recap}

\begin{frame}
	\frametitle{Relevance}
	\begin{center}
		\vfill
		Customers care about their services being up, not about individual service components
		\vfill
		Discern between primary (service-relevant) and secondary (informational / debugging) SLIs; alert only on the former
		\vfill
		Anything currently or imminently impacting customer service must be alerted upon.
		\vfill
	\end{center}
\end{frame}

\begin{frame}
	\frametitle{Containment}
	\begin{center}
		\vfill
		Service delineations are the perfect boundaries for containing complexity
		\vfill
	\end{center}
\end{frame}

%\begin{frame}
%	\frametitle{Dependence}
%	\begin{center}
%		\vfill
%		One services' primary SLI is the depending services' secondary SLI
%		\vfill
%	\end{center}
%\end{frame}


\subsection{Bringing it all together}

\begin{frame}
	\frametitle{What's all this, now?}
	\begin{center}
		\vfill
		Monitoring tells you whether the system works.\\
		Observability lets you ask why it’s not working.\\
		-\\
		\textit{Baron Schwartz}
		\vfill
	\end{center}
\end{frame}

\begin{frame}
	\frametitle{}
	\begin{center}
		\vfill
		Observability is not something you ever achieve, you can always improve on it.\\
		\vfill
		As such there's not the One True Thing to do, it's about establishing the correct mindset.
		\vfill
	\end{center}
\end{frame}

%\begin{frame}
%	\frametitle{}
%	\begin{center}
%		\vfill
%		Monitoring tells you whether the system works. Observability lets you ask why it’s not working.\\
%		Baron Schwartz
%		\vfill
%	\end{center}
%\end{frame}

\begin{frame}
	\frametitle{BCPs}
	\begin{itemize}
		\item Every outage gets a blame-free(!) post-mortem; and this includes a review of all relevant SLI \& SLO
		\begin{itemize}
			\item ..are they still useful?
			\item ..would you have been quicker if you would have had different/more data?
			\item ..should you retire some data collection?
		\end{itemize}
		\item Link services together in your dashboards, etc
		\begin{itemize}
			\item Make jumping into underlying services and their data as fluent as possible
			\item Surface important insights from underlying services as context
		\end{itemize}
	\end{itemize}
\end{frame}

\begin{frame}
	\frametitle{BCPs}
	\begin{itemize}
		\item Avoid relying only on blackbox data where possible; you need to get into your systems and extract fine-grained and meaningful data
		\begin{itemize}
			\item Best case, this means instrumenting your code to extract metrics and traces\\
				Every time you are even considering to place a DEBUG statement into a codepath, put a counter
			\item In the networking space, this often means requiring better data from your vendors.\\
				Explain what and why you need it, then force them via conditional POs, etc.
		\end{itemize}
	\end{itemize}
\end{frame}

\begin{frame}
	\frametitle{BCPs}
	\begin{itemize}
		\item You (hopefully) know your services best, so create debugging stories in advance
		\begin{itemize}
			\item What are common and/or critical paths while getting visibility into issues?
			\item Which parts of these paths can you automate (more)?
			\item Does it makes sense to introduce new (internal) service boundaries and/or contracts to create new compartments?
		\end{itemize}
		\item ELI5: Find an non-expert and explain your services to them
	\end{itemize}
\end{frame}

\begin{frame}
	\frametitle{BCPs}
	\begin{itemize}
		\item Collecting 10x or 100x more data means you have more substance to work with
		\item Avoid data lakes, attach meaningful metadata as early as possible
		\item Your tools must be able to handle this load
		\item Even more important, they must make handling the amounts of data manageable, and support automation
	\end{itemize}
\end{frame}

% low-effort data collection with modern tools

\section{Outro}

\begin{frame}
	\frametitle{Thanks!}
	\begin{center}
		\vfill
		Thanks for listening!\\
		\vfill
		Questions?
		\vfill
		Email me if you want a job in Munich.
		\vfill
		See slide footer for contact info.
		\vfill
	\end{center}
\end{frame}

\end{document}

%\begin{frame}
%	\frametitle{}
%	\begin{itemize}
%		\item 
%		\item 
%		\item 
%		\item 
%		\item 
%	\end{itemize}
%\end{frame}
