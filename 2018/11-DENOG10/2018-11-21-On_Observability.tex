\documentclass[t]{beamer}
\usepackage{helvet}
\usepackage{calc}
\usepackage[utf8]{inputenc}
\usepackage[english]{babel}

\usetheme{Ilmenau}

\setbeamercovered{transparent}
\setbeamertemplate{navigation symbols}{}

\usepackage{units}
\usepackage{amsbsy}
\usepackage{amsmath}
\usepackage{amssymb}
\usepackage{graphics}
\usepackage{graphicx}
\usepackage{epsf}
\usepackage{epsfig}
\usepackage{fixmath}
%\usepackage{pgfmath}
\usepackage{wrapfig}


\title{On Observability}
\subtitle{}
\author{Richard Hartmann,\\
RichiH@\{freenode,OFTC,IRCnet\},\\
richih@\{debian,fosdem,richih\}.org,\\
@TwitchiH}
\date{2018-11-XX}


\begin{document}

% hide all subsections
\setcounter{tocdepth}{1}

\section{Introduction}

\subsection{}

\begin{frame}
	\titlepage
\end{frame}

%\begin{frame}
%	\frametitle{Statistics}
%	\tableofcontents
%\end{frame}

\subsection{}

\begin{frame}
	\frametitle{`whoami`}
	\begin{itemize}
		\item Richard "RichiH" Hartmann
		\item Swiss army chainsaw at SpaceNet
		\begin{itemize}
			\item Leading the build of one of the most modern datacenters in Europe
			\item ...and always looking for nice co-workers in the Munich area
		\end{itemize}
		\item FOSDEM, DebConf, DENOGx, PromCon staff
		\item Author of \url{https://github.com/RichiH/vcsh}
		\item Debian Developer
		\item Prometheus team member
		\item OpenMetrics founder
	\end{itemize}
\end{frame}


\section{Intro}

\subsection{Definitions}

\begin{frame}
	\frametitle{Buzzword}
	\begin{center}
		\vfill
		buzzword, n:\\
		A useful concept which has been picked up by everyone without understanding its deeper meaning and used so often that it's devoid of its original context and defintion.
		May revert to usefulness in the same or different meaning, or die off.
		\vfill
	\end{center}
\end{frame}

\begin{frame}
	\frametitle{Cargo culting}
	\begin{center}
		\vfill
		cargo culting, v\\
		Villagers on remote Pacific islands observed U.S. soldiers building marker fires and runways during WWII; this made planes come and bring gifts from the heavens.
		Cults emerged which builit bonfires and runways in the hopes of getting more gifts.\\
		Also see: copy \& paste
		\vfill
	\end{center}
\end{frame}

\begin{frame}
	\frametitle{Monitoring}
	\begin{center}
		\vfill
		monitoring, n\\
		Old buzzword\\
		Too often: focus is put on collecting, persisting, and alerting on just any data, as long as its data
		\vfill
	\end{center}
\end{frame}

\begin{frame}
	\frametitle{Observability}
	\begin{center}
		\vfill
		observability, n\\
		Function of a system in which humans and machines can observe, understand, and act on the state of said system
		\vfill
	\end{center}
\end{frame}

\begin{frame}
	\frametitle{Thanks!}
	\begin{center}
		\vfill
		Thanks for listening!\\
		\vfill
		Questions?
		\vfill
		Email me if you want a job in Munich.
		\vfill
		See slide footer for contact info.
		\vfill
	\end{center}
\end{frame}


\subsection{Outlook}

\begin{frame}
	\frametitle{Learnings}
	\vfill
	\begin{itemize}
		\item TODO enumerate %% TODO enumerate

		\item Baseline of monitoring
		\item Types of monitoring data and when to use them

		\item Service, contracts, SL\{I,O,A\}, etc
		\item Services upon services

		\item Layers upon layers, containing complexity, types of completixy
		\item Bringing it all together
	\end{itemize}
	\vfill
\end{frame}



\section{Monitoring}


\subsection{Baseline of monitoring}

\begin{frame}
	\frametitle{Recap}
	\begin{center}
		\vfill
		Monitoring is the bedrock of everything (in IT)
		\vfill
		Hope is not a strategy
		\vfill
	\end{center}
\end{frame}

\begin{frame}
	\frametitle{Claim}
	\begin{center}
		\vfill
		Uninformed, or cargo culted, monitoring equals hope\\
		(see: ISO 9001 \& 27001)
		\vfill
		So we need informated decisions, made on a factual basis
		\vfill
	\end{center}
\end{frame}


\subsection{Types of monitoring data}

\begin{frame}
	\frametitle{50:50}
	\begin{center}
		\vfill
		Broadly speaking, there are metrics and events
		\vfill
		Metrics: Changes over time
		\vfill
		Events: Specific points in time
		\vfill
	\end{center}
\end{frame}


\subsection{Metrics}

\begin{frame}
	\frametitle{Metrics}
	\begin{itemize}
		\item Numerical data
		\begin{itemize}
			\item Counters: Things going up monotinically, e.g. total transmitted bytes
			\item Gauges: Things going up and down, e.g. tempratures
			\item Histograms aka percentiles: Things going into buckets, e.g. latency
		\end{itemize}
		\item Counters and histograms lose, or compress, data (in the common case)
		\item Easy to handle at scale
		\item You can do math with them!
	\end{itemize}
\end{frame}


\subsection{Events}

\begin{frame}
	\frametitle{Logs}
	\begin{itemize}
		\item Text items
		\item Usually with inlined metadata
		\item Scale linearly with service load
		\item Can be summarized into counters
		\item Special case: compliance \& due dilligence
	\end{itemize}
\end{frame}

\begin{frame}
	\frametitle{Traces}
	\begin{itemize}
		\item Execution path along the, hopefully annotated, code
		\item Impacts code runtime, aka expensive
		\item Can hide race conditions and other timing-dependent issues
		\item Usually disabled or sampled
	\end{itemize}
\end{frame}

\begin{frame}
	\frametitle{Dumps}
	\begin{itemize}
		\item Thrown when program aborts abnormally
		\item Execution path along the code
		\item Not annotated unless compiler artefacts of the exact same program are available
		\item You want to avoid them, but you also want to collect them if they happen
	\end{itemize}
\end{frame}



\section{Services}


\subsection{Baseline of services}

\begin{frame}
	\frametitle{What's a service?}
	\begin{center}
		\vfill
		A service is anything a different entity relies upon
		\vfill
		This entity might be another team, a customer, or yourself
		\vfill
	\end{center}
\end{frame}

\begin{frame}
	\frametitle{Handover}
	\begin{center}
		\vfill
		Service delinations have many names: interface, API, contract
		\vfill
		I like to think of all of them as contracts, but why?
		\vfill
	\end{center}
\end{frame}


\subsection{Pop culture references}

\begin{frame}
	\frametitle{Tetris}
	\begin{center}
		\vfill
		Services build on top of each other
		\vfill
		(Network * x + machine/container/kubelet * y + daemon/microservice * z) * n = HTTP service
		\vfill
	\end{center}
\end{frame}

\begin{frame}
	\frametitle{Jenga}
	\begin{center}
		\vfill
		This tower can topple if the underlying building blocks are removed without due consideration
		\vfill
		Contract implies a firm commitment, which is why I like this term
		\vfill
	\end{center}
\end{frame}

\begin{frame}
	\frametitle{Bowling}
	\begin{center}
		\vfill
		This tower can topple if the underlying building blocks are removed 
		\vfill
		"Contract" implies a firm commitment which can only be changed if all parties agree.\\
		This is why I like this term
		\vfill
	\end{center}
\end{frame}

\begin{frame}
	\frametitle{Networking}
	\begin{center}
		\vfill
		There's another common term for contract: layer
		\vfill
		Imagine if someone simply changed how IP works
		\vfill
	\end{center}
\end{frame}

\begin{frame}
	\frametitle{Trolling}
	\begin{center}
		\vfill
		For example, someone could simply claim that IP addresses have 128 instead of 32 bits all of sudden...
		\vfill
	\end{center}
\end{frame}

\begin{frame}
	\frametitle{Recap}
	\begin{center}
		\vfill
		So we agree that layering makes sense, but why do we agree?
		\vfill
	\end{center}
\end{frame}

\begin{frame}
	\frametitle{It's complicated}
	\begin{center}
		\vfill
		We do this to contain system-inherent complexity
		\vfill
	\end{center}
\end{frame}

\begin{frame}
	\frametitle{It's complicated}
	\begin{center}
		\vfill
		A CPU is highly complex, but we are happy to trust their hidden complexity because there's a well-defined service boundary %%%%TODO
		\vfill
	\end{center}
\end{frame}

%		\item Service, contracts, SL\{I,O,A\}, etc
%		\item Services upon services
%		\item Layers upon layers, containing complexity, types of completixy
%		\item Bringing it all together
% TODO what monitoring data is useful for what?
%TODO every time you consider to use a DEBUG statement, toss a counter in there

\section{Outro}

\begin{frame}
	\frametitle{Thanks!}
	\begin{center}
		\vfill
		Thanks for listening!\\
		\vfill
		Questions?
		\vfill
		Email me if you want a job in Munich.
		\vfill
		See slide footer for contact info.
		\vfill
	\end{center}
\end{frame}

\end{document}

%\begin{frame}
%	\frametitle{}
%	\begin{itemize}
%		\item 
%		\item 
%		\item 
%		\item 
%		\item 
%	\end{itemize}
%\end{frame}
