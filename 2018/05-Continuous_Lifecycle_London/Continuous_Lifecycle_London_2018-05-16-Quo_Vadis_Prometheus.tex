\documentclass[t]{beamer}
\usepackage{helvet}
\usepackage{calc}
\usepackage[utf8]{inputenc}
\usepackage[english]{babel}

\usetheme{Ilmenau}

\setbeamercovered{transparent}
\setbeamertemplate{navigation symbols}{}

\usepackage{units}
\usepackage{amsbsy}
\usepackage{amsmath}
\usepackage{amssymb}
\usepackage{graphics}
\usepackage{graphicx}
\usepackage{epsf}
\usepackage{epsfig}
\usepackage{fixmath}
%\usepackage{pgfmath}
\usepackage{wrapfig}


\title{Quo vadis, Prometheus?}
\subtitle{Monitoring. At scale.}
\author{Richard Hartmann,\\
RichiH@\{freenode,OFTC,IRCnet\},\\
richih@\{fosdem,debian,richih\}.org,\\
richard.hartmann@space.net}
\date{2018-05-16}


\begin{document}

% hide all subsections
\setcounter{tocdepth}{1}

\section{Introduction}

\subsection{}

\begin{frame}
	\titlepage
\end{frame}

%\begin{frame}
%	\frametitle{Statistics}
%	\tableofcontents
%\end{frame}

\subsection{}

\begin{frame}
	\frametitle{`whoami`}
	\begin{itemize}
		\item Richard "RichiH" Hartmann
		\item Swiss army chainsaw at SpaceNet
		\begin{itemize}
			\item Currently responsible for building one of the most modern datacenters in Europe
			\item ...and always looking for nice co-workers in the Munich area
		\end{itemize}
		\item FOSDEM, DebConf, DENOGx, PromCon staff
		\item Author of \url{https://github.com/RichiH/vcsh}
		\item Debian Developer
		\item Prometheus team member
	\end{itemize}
\end{frame}


\section{Quick intro}

\subsection{Do we even need this section?}

% API commitments
% one exporter per service
% focus on services
% failures happen at scale

% HA? run two
% federation
% kein auth/encryption -> reverse proxy


\begin{frame}
	\frametitle{Show of hands}
	\begin{itemize}
		\item Who has heard of Prometheus?
		\item Who is considering using Prometheus?
		\item Who is POCing Prometheus?
		\item Who uses Prometheus in production?
	\end{itemize}
\end{frame}

\begin{frame}
	\frametitle{Prometheus 101}
	\begin{itemize}
		\item Inspired by Google's Borgmon
		\item Time series database
		\item int64 timestamp, float64 value
		\item Ecosystem of instrumentation \& exporters
		\item Not for events
		\begin{itemize}
			\item Logging
			\item Tracing (more on that later)
			\item etc.
		\end{itemize}
		\item Dashboarding via Grafana
	\end{itemize}
\end{frame}

\begin{frame}
	\frametitle{Main selling points}
	\begin{itemize}
		\item Highly dynamic, built-in service discovery
		\item No hierarchical model, n-dimensional label set
		\item PromQL: for processing, graphing, alerting, and export
		\item Simple operation
		\item Highly efficient
	\end{itemize}
\end{frame}

\begin{frame}
	\frametitle{Cloudy with a chance of buzzwords}
	\begin{itemize}
		\item So it's built with highly dynamic environments in mind
		\item It's the second project to ever join CNCF and the de facto standard in cloud-native monitoring
		\item Kubelets, sidecars, microservices, ALL the cloud-native
		\vfill
		\item But it's a monolithic application
		\vfill
		\item ...why?
	\end{itemize}
\end{frame}

\begin{frame}
	\frametitle{Resilience, resilience, and also resilience}
	\begin{itemize}
		\item What do you need for operations?
		\begin{itemize}
			\item Power and cooling
			\item Network connectivity
			\item Monitoring
		\end{itemize}
		\item The rest you can fix
	\end{itemize}
\end{frame}


\section{2.0 to 2.2.1}
\subsection{}

\begin{frame}
	\frametitle{Three main features}
	\begin{itemize}
		\item Storage backend
		\begin{itemize}
			\item Caveat: Prometheus 2.0 comes with storage v3
		\end{itemize}
		\item Staleness handling
		\item Remote read \& write API is now stable-ish
		\item Links to in-depth talks about these features are at the end
	\end{itemize}
\end{frame}


\subsection{Storage}

\begin{frame}
	\frametitle{Prometheus 1.x}
	\begin{itemize}
		\item We used to have one file per time series
		\item Relatively easy to implement
		\item Pretty efficient
		\item Why change?
	\end{itemize}
\end{frame}

\begin{frame}
	\frametitle{Churn}
	\begin{itemize}
		\item Churn was becoming more and more of a problem
		\item There's a company with a 15 minute maximum lifetime for their containers
		\item If you have a lot of files which might contain data for any given time frame, you need to look at all of them
	\end{itemize}
\end{frame}

%\begin{frame}
%	\frametitle{Storage v3}
%	\begin{itemize}
%		\item Fabian Reinartz had an idea about new storage
%		\item This POC turned out to be so good, we decided to cut a major release for it
%		\item How does it work?
%	\end{itemize}
%\end{frame}

\begin{frame}
	\frametitle{One file per series}
	\includegraphics[width=\textwidth]{storage--file_per_series.png}
\end{frame}

\begin{frame}
	\frametitle{Selection}
	\includegraphics[width=\textwidth]{storage--file_per_series_with_selection.png}
\end{frame}

\begin{frame}
	\frametitle{Blocks}
	\includegraphics[width=\textwidth]{storage--block_with_selection.png}
\end{frame}

%\begin{frame}
%	\frametitle{Storage v3}
%	\begin{itemize}
%		\item Deletions set tombstones
%		\item Actual deletion done via compaction runs, or triggered by large tombstone ratio
%		\item Oh, and we now simply mmap storage blocks into RAM and let the OS handle the rest
%	\end{itemize}
%\end{frame}

\begin{frame}
	\frametitle{Test setup}
	\begin{itemize}
		\item Kubernetes cluster with dedicated Prometheus nodes
		\item 800 microservice instances and Kubernetes components
		\item 120k samples/sec
		\item 300k active time series
		\item Swap out 50\% of all pods every 10 minutes
	\end{itemize}
\end{frame}

\begin{frame}
	\frametitle{Results}
	\begin{itemize}
		\item 15x reduction in memory usage
		\item 6x reduction in CPU usage
		\item 80-100x reduction in disk writes
		\item 5x reduction in on-disk size
		\item 4x reduction in query latency on expensive queries
		\item Want to reproduce? \url{https://github.com/prometheus/prombench}
	\end{itemize}
\end{frame}

%%%%%%%%%%%%%%%%%%%\begin{frame}
%%%%%%%%%%%%%%%%%%%	\frametitle{Other 2.x news}
%%%%%%%%%%%%%%%%%%%	\begin{itemize}
%%%%%%%%%%%%%%%%%%%		\item Snapshotting and backing up Prometheus works
%%%%%%%%%%%%%%%%%%%		\item New staleness handling
%%%%%%%%%%%%%%%%%%%	\end{itemize}
%%%%%%%%%%%%%%%%%%%\end{frame}


\subsection{Staleness}


\begin{frame}
	\frametitle{Downside of handling churn}
	\begin{itemize}
		\item So now we can handle extreme churn
		\item ...and suddenly, five minutes staleness timeouts seem awfully long
		\begin{itemize}
			\item Down alerts continue to fire
			\item Double counting
			\item Other icky corner cases
		\end{itemize}
	\end{itemize}
\end{frame}

\begin{frame}
	\frametitle{Results}
	\begin{itemize}
		\item When a target goes away, its time series are considered stale
		\item When a target no longer returns a time series, it is considered stale
		\item Longer eval/scrape intervals are now easier to handle
	\end{itemize}
\end{frame}

%\begin{frame}
%	\frametitle{The way there}
%	\begin{itemize}
%		\item The actual technical background is too complex for this talk
%		\item Find recording and slides by Brian Brazil at the end of this talk
%	\end{itemize}
%\end{frame}


\subsection{Remote read API}

\begin{frame}
	\frametitle{Playing nicely with others}
	\begin{itemize}
		\item We now have a stable-ish remote read/write API
		\item Which we're already using ourselves; it's the recommended upgrade path from 1.x
		\begin{itemize}
			\item You need to upgrade to 1.8.2 or later for this to work
		\end{itemize}
	\end{itemize}
\end{frame}


\subsection{Downsides..}

\begin{frame}
	\frametitle{So, about backfill and explicit timestamps...}
	\begin{itemize}
		\item If explicit timestamps were icky before, this has now become worse
		\item You can not ingest data older than the age of the current storage block, nor data much newer
		\item Staleness vs timestamps is non-trivial
	\end{itemize}
\end{frame}



\section{2.3 to 3.x}


\subsection{ACID}

\begin{frame}
	\frametitle{ACID databases...}
	\begin{itemize}
		\item \textbf{A}tomicity - since 1.x
		\item \textbf{C}onsistency - since 1.x
		\item \textbf{I}solation - will happen within 2.x
		\item \textbf{D}urability - since 2.0
	\end{itemize}
\end{frame}

\begin{frame}
	\frametitle{Isolation}
	\begin{itemize}
		\item Each append action gets a write ID (64 bit monotonic counter)
		\item Every sample's write ID is noted along with value and timestamp
		\item Any append action which has not yet committed or rolled back is ignored at query time
		\item We keep write IDs in memory; if we restart or crash, the atomicity of the write ahead log will protect us
	\end{itemize}
\end{frame}

\subsection{PromQL}

\begin{frame}
	\frametitle{Quick is not quick enough}
	\begin{itemize}
		\item Brian Brazil is currently working on optimizing PromQL
		\item 5x faster for time vector functions
		\item 100x reduction in garbage to collect
		\item \url{https://github.com/prometheus/prometheus/pull/3966}
	\end{itemize}
\end{frame}

\subsection{Stability}

\begin{frame}
	\frametitle{Release stability}
	\begin{itemize}
		\item Every single release since 2.0.0 has had issues
		\item Some bugs and some human mistakes in the release process
		\item 2.2.1 is clean and stable
		\item Always running latest is the cloud-native approach, but this is still not acceptable
		\item We put in more checks and balances to ensure cleaner releases going forward
	\end{itemize}
\end{frame}

\begin{frame}
	\frametitle{Security \& quality}
	\begin{itemize}
		\item CNCF is sponsoring external code review for all member projects
		\item Focussed on security, but this always means looking at stability as well
		\item Keep in mind that Prometheus willfully ignores most security considerations
		\item Encryption, authentication, and authorization should be handled via reverse proxies etc
	\end{itemize}
\end{frame}


\subsection{Long-term storage}

\begin{frame}
	\frametitle{Problem statement}
	\begin{itemize}
		\item Long-term storage is the last remaining major feature left untackled
		\item Fundamentally, Prometheus operates as distinct data islands
		\item As there's no backfill, data dies along with its instance by default
	\end{itemize}
\end{frame}

\begin{frame}
	\frametitle{Solutions}
	\begin{itemize}
		\item Storage v3 supports backups efficiently and effectively
		\item Remote read-write allows you to integrate with a growing list of projects and products, e.g. Cortex
		\item On storage level, there are object storage backends for Prometheus, e.g. Thanos
		\vfill
		\item We deliberately do no endorse any particular approach or solution; this might change over time
	\end{itemize}
\end{frame}

\section{Beyond}
\subsection{OpenMetrics}

\begin{frame}
	\frametitle{Humble aspirations}
	\begin{itemize}
		\item When we say that we want to change how the world does monitoring, we mean it
		\item One of our most powerful features are labels
		\item Labels are encoded in our exposition format
		\item Some third-party projects and vendors have an issue with supporting a "competing" project
	\end{itemize}
\end{frame}

\begin{frame}
	\frametitle{What do?}
	\begin{itemize}
		\item We are spinning out Prometheus' exposition format
		\item Face-to-face kick-off last August at Google London
		\item Independent CNCF member project, including IETF RFC, etc
		\item We finished our technical discussions last week and I am currently writing the Internet Draft
		\item \url{https://github.com/RichiH/OpenMetrics}
	\end{itemize}
\end{frame}

\begin{frame}
	\frametitle{Beyond metrics}
	\begin{itemize}
		\item OpenMetrics supports more than just metrics
		\item Every single data point in a time series can point to one single event
		\item Especially useful if you emit one trace id per histogram bucket
		\item Some integrations already support this concept, e.g. OpenCensus
		\item Ingestors are free to discard this optional data, e.g. Prometheus
	\end{itemize}
\end{frame}

\begin{frame}
	\frametitle{Bringing observability together}
	\begin{itemize}
		\item OpenTracing already on board with this effort
		\item Will hammer out more details and have a face-to-face during KubeCon Seattle this December
		\item Will most likely have a three-day side track called Observacon
		\item Long-term goal is one common, modular, well-engineered standard under a new name
	\end{itemize}
\end{frame}



\begin{frame}
	\frametitle{First committers to adopt, too many to list all}
	\begin{itemize}
		\item Cloudflare
		\item CNCF at large
		\item GitLab
		\item Google
		\item Grafana
		\item InfluxData
		\item Kausal.co
		\item Oath.com / Yahoo / Verizon
		\item RobustPerception
		\item SpaceNet
		\item Uber
	\end{itemize}
\end{frame}


\subsection{Face-to-face}

\begin{frame}
	\frametitle{PromCon 2018}
	\begin{itemize}
		\item August 09 \& 10
		\item Held at Google office in Munich
		\item Organized by the Prometheus community, mainly be me
		\item Sponsorship still open
		\item CfP still open for two weeks
		\item Diversity funding available: for both ticket and travel
	\end{itemize}
\end{frame}


\subsection{Long term promises}

\begin{frame}
	\frametitle{Generally speaking...}
	\begin{itemize}
		\item Yes, we want to change the world
		\item Simple and resilient operation of Prometheus remains a core goal
		\item More project and software integrations... and we're talking to hardware vendors as well
		\item Supporting tomorrow's 10x scale today
	\end{itemize}
\end{frame}



\section{Outro}


\subsection{}

\begin{frame}
	\frametitle{Relevant talks}
	\begin{itemize}
		\item Storing 16 Bytes at Scale: \url{https://promcon.io/2017-munich/talks/staleness-in-prometheus-2-0/}
		\item Staleness and Isolation in Prometheus 2.0: \url{https://promcon.io/2017-munich/talks/staleness-in-prometheus-2-0/}
		\item Social aspects of change: \url{https://promcon.io/2017-munich/talks/social-aspects-of-change/}
	\end{itemize}
\end{frame}


\begin{frame}
	\frametitle{Further reading}
	\begin{itemize}
		\item Prometheus 2017 Dev Summit: \url{https://docs.google.com/document/d/1DaHFao0saZ3MDt9yuuxLaCQg8WGadO8s44i3cxSARcM/edit}
		\item My other talks: \url{https://github.com/RichiH/talks/}
		\item OpenMetrics: \url{https://github.com/RichiH/OpenMetrics}
	\end{itemize}
\end{frame}

\begin{frame}
	\frametitle{Thanks!}
		\begin{center}
			\vfill
			Thanks for listening!\\
			\vfill
			Questions?
			\vfill
			Email me if you want a job in Munich.
			\vfill
			See slide footer for contact info.
			\vfill
		\end{center}
\end{frame}

\end{document}

%\begin{frame}
%	\frametitle{}
%	\begin{itemize}
%		\item 
%		\item 
%		\item 
%		\item 
%		\item 
%	\end{itemize}
%\end{frame}
