\documentclass[t]{beamer}
\usepackage{helvet}
\usepackage{calc}
\usepackage[utf8]{inputenc}
\usepackage[english]{babel}

\usetheme{Ilmenau}

\setbeamercovered{transparent}
\setbeamertemplate{navigation symbols}{}

\usepackage{units}
\usepackage{amsbsy}
\usepackage{amsmath}
\usepackage{amssymb}
\usepackage{graphics}
\usepackage{graphicx}
\usepackage{epsf}
\usepackage{epsfig}
\usepackage{fixmath}
%\usepackage{pgfmath}
\usepackage{wrapfig}


\title{vcsh}
\subtitle{manage config files in \$HOME via fake bare git repositories}
\author{Richard Hartmann,\\
RichiH@\{freenode,OFTC,IRCnet\},\\
richih.mailinglist@gmail.com}
\date{2012-02-04}



\begin{document}


\begin{frame}
	\titlepage
\end{frame}

\begin{frame}
	\frametitle{Outline}
	\tableofcontents
\end{frame}


\section{Intro}

\begin{frame}
	\frametitle{Outline}
	\tableofcontents[currentsection]
\end{frame}

\begin{frame}
	\frametitle{Who am I?}
	\begin{itemize}
		\item Project \& Network Operations Manager at Globalways AG
		\item freenode \& OFTC staff
		\item Passionate about FLOSS
		\item Author of vcsh
	\end{itemize}
\end{frame}

\begin{frame}
	\frametitle{What is git?}
	\begin{itemize}
		\item Version control system
		\item Distributed
		\begin{itemize}
			\item No need for central repository
			\item Allows you to commit while offline
		\end{itemize}
		\item Full history in every checkout
		\item Best version control system available (imo...)
	\end{itemize}
\end{frame}


\section{Technical details}

\begin{frame}
	\frametitle{Outline}
	\tableofcontents[currentsection]
\end{frame}

\begin{frame}
	\frametitle{What is vcsh?}
	\begin{itemize}
		\item Based on git
		\item git unable to maintain several working copies in one directory
		\item Sucks if you want to keep your configs in git
		\item vcsh uses fake bare git repositories to work around this
	\end{itemize}
\end{frame}

\begin{frame}
	\frametitle{fake bare.. what?}
	\begin{itemize}
		\item Normal git repo:
		\begin{itemize}
			\item git data in \texttt{\$GIT\_DIR}
			\item working copy in \texttt{\$GIT\_WORK\_TREE}
			\item \texttt{\$GIT\_WORK\_TREE/.git == \$GIT\_DIR}
		\end{itemize}
		\item Bare git repo:
		\begin{itemize}
			\item git data in \texttt{\$GIT\_DIR}
			\item no  \texttt{\$GIT\_WORK\_TREE}
		\end{itemize}
		\item Fake bare git repo:
		\begin{itemize}
			\item git data in \texttt{\$GIT\_DIR}
			\item working copy in \texttt{\$GIT\_WORK\_TREE}
			\item \texttt{\$GIT\_DIR == \$XDG\_CONFIG\_HOME/vcsh/repo.d/\$repo.vcsh}
			\item \texttt{\$GIT\_WORK\_TREE == \$HOME}
		\end{itemize}
	\end{itemize}
\end{frame}

\begin{frame}
	\frametitle{Why use vcsh?}
	\begin{itemize}
		\item Several \texttt{\$GIT\_WORK\_TREE} in \texttt$HOME}
		\item 
		\item 
		\item 
		\item 
	\end{itemize}
\end{frame}

%%%%%%%%%%%%%%%%%%%%%%%%%%%%%%%%%%%%%%%%%%%%%%%%%%%%%%%%%%%%%%%%%%%%%%%%%%%%%%%%%%%%%%%%%%

\section{Using vcsh}

\begin{frame}
	\frametitle{Outline}
	\tableofcontents[currentsection]
\end{frame}

\begin{frame}
	\frametitle{Cloning a repo}
	\begin{itemize}
		\item 
		\item 
		\item 
		\item 
		\item 
	\end{itemize}
\end{frame}

% mr integration
% zsh integration
% git-annex integration

%%%%%%%%%%%%%%%%%%%%%%%%%%%%%%%%%%%%%%%%%%%%%%%%%%%%%%%%%%%%%%%%%%%%%%%%%%%%%%%%%%


\section{Outlook}

\begin{frame}
	\frametitle{Outline}
	\tableofcontents[currentsection]
\end{frame}

\begin{frame}
	\frametitle{Future work}
	\begin{itemize}
		\item More unit tests
		\item Get vcsh into more distributions
		\item Spread awareness to reach critical mass
		\item Maybe extend support to subversion, mercurial, etc
	\end{itemize}
\end{frame}

\section{Outro}

\begin{frame}
	\frametitle{Outline}
	\tableofcontents[currentsection]
\end{frame}

\begin{frame}
	\frametitle{Where to get it}
	\begin{itemize}
		\item \texttt{git clone git://github.com/RichiH/vcsh.git}
		\item Native packages for
		\begin{itemize}
			\item Debian
			\item Ubuntu
			\item Arch Linux
		\end{itemize}
		\item Small bug in README.md, use v0.20120203 or git
	\end{itemize}
\end{frame}

\begin{frame}
	\frametitle{Thanks!}
		\begin{center}
			\vfill
			Thanks for listening!\\
			\vfill
			Questions? Follow me outside when my time-slot is over.
			\vfill
			See slide footer for further contact Information.
			\vfill
		\end{center}
\end{frame}




\end{document}


%\begin{frame}
%	\frametitle{}
%	\begin{itemize}
%		\item 
%		\item 
%		\item 
%		\item 
%		\item 
%	\end{itemize}
%\end{frame}
