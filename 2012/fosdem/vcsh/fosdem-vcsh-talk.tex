\documentclass[t]{beamer}
\usepackage{helvet}
\usepackage{calc}
\usepackage[utf8]{inputenc}
\usepackage[english]{babel}

\usetheme{Ilmenau}

\setbeamercovered{transparent}
\setbeamertemplate{navigation symbols}{}

\usepackage{units}
\usepackage{amsbsy}
\usepackage{amsmath}
\usepackage{amssymb}
\usepackage{graphics}
\usepackage{graphicx}
\usepackage{epsf}
\usepackage{epsfig}
\usepackage{fixmath}
%\usepackage{pgfmath}
\usepackage{wrapfig}


\title{vcsh}
\subtitle{manage config files in \$HOME via fake bare git repositories}
\author{Richard Hartmann,\\
RichiH@\{freenode,OFTC,IRCnet\},\\
richih.mailinglist@gmail.com}
\date{2012-02-04}



\begin{document}


\begin{frame}
	\titlepage
\end{frame}

\begin{frame}
	\frametitle{Outline}
	\tableofcontents
\end{frame}


\section{Intro}

\begin{frame}
	\frametitle{Outline}
	\tableofcontents[currentsection]
\end{frame}

\begin{frame}
	\frametitle{Who am I?}
	\begin{itemize}
		\item Project \& Network Operations Manager at Globalways AG
		\item freenode \& OFTC staff
		\item Passionate about FLOSS
		\item Author of vcsh
	\end{itemize}
\end{frame}

\begin{frame}
	\frametitle{What is git?}
	\begin{itemize}
		\item Version control system
		\item Distributed
		\begin{itemize}
			\item No need for central repository
			\item Allows you to commit while offline
		\end{itemize}
		\item Full history in every checkout
		\item Best version control system available (imo...)
	\end{itemize}
\end{frame}


\section{Technical details}

\begin{frame}
	\frametitle{Outline}
	\tableofcontents[currentsection]
\end{frame}

\begin{frame}
	\frametitle{What is vcsh?}
	\begin{itemize}
		\item Implemented in POSIX shell; portable
		\item "version control shell" or "version control system \texttt{\$HOME}"
		\item Based on git
		\begin{itemize}
			\item git unable to maintain several working copies in one directory
			\item Sucks if you want to keep your configs in git
		\end{itemize}
		\item vcsh uses fake bare git repositories to work around this
		\item Think of it as an extension to git
	\end{itemize}
\end{frame}

\begin{frame}
	\frametitle{fake bare.. what?}
	\begin{itemize}
		\item Normal git repo:
		\begin{itemize}
			\item working copy in \texttt{\$GIT\_WORK\_TREE}
			\item git data in \texttt{\$GIT\_WORK\_TREE/.git} aka \texttt{\$GIT\_DIR}
		\end{itemize}
		\item Bare git repo:
		\begin{itemize}
			\item git data in \texttt{\$GIT\_DIR}
			\item no  \texttt{\$GIT\_WORK\_TREE}
		\end{itemize}
		\item Fake bare git repo:
		\begin{itemize}
			\item working copy in \texttt{\$GIT\_WORK\_TREE}
			\item git data in \texttt{\$GIT\_DIR}
			\item \texttt{\$GIT\_WORK\_TREE == \$HOME}
			\item \texttt{\$GIT\_DIR == \$XDG\_CONFIG\_HOME/vcsh/repo.d/\$repo.vcsh}
			\item \texttt{core.bare = false}
		\end{itemize}
	\end{itemize}
\end{frame}

\begin{frame}
	\frametitle{Problems with fake bare git repos}
	\begin{itemize}
		\item Fake bare repos are messy to set up and use
		\item Reason why git disallows shared \texttt{\$GIT\_WORK\_TREE}: complexity due to context-dependency
		\item Mistakes lead to confusion or data loss; imagine \texttt{\$GIT\_WORK\_TREE} set and
		\begin{itemize}
			\item \texttt{git add}
			\item \texttt{git reset --hard HEAD\~{}1}
			\item \texttt{git checkout -- *}
			\item \texttt{git clean -f}
		\end{itemize}
	\end{itemize}
\end{frame}

\begin{frame}
	\frametitle{Solution: vcsh}
	\begin{itemize}
		\item Wraps around git
		\item Hides complexity and does sanity checks
		\item Several git repos checked out into \texttt{\$HOME} at once
		\begin{itemize}
			\item One repo for zsh, vim, mplayer, etc
			\item Enables specific subsets of repos per host
		\end{itemize}
		\item Manages complete repo life-cycle
	\end{itemize}
\end{frame}

\section{Using vcsh}

\begin{frame}
	\frametitle{Outline}
	\tableofcontents[currentsection]
\end{frame}

\begin{frame}
	\frametitle{Create new repo}
	\texttt{ \\
		\# create new repo \\
		vcsh init vim \\
		\# add files to it \\
		vcsh run vim git add .vim .vimrc \\
		\# commit using shorthand form \\
		vcsh vim commit \\
		\# push using longhand form \\
		vcsh run vim git push
	}
\end{frame}

\begin{frame}
	\frametitle{Made-up life-cycle}
	\texttt{ \\
		\# clone repo into new name zsh \\
		vcsh clone git://github.com/RichiH/zshrc.git zsh \\
		\# optionally update legacy repos \\
		vcsh setup zsh \\
		\# display all files managed by this repo \\
		vcsh run zsh git ls-files \\
		\# rename repo just because \\
		vcsh rename zsh zshrc \\
		\# delete repo \\
		vcsh delete zshrc
	}
\end{frame}

\begin{frame}
	\frametitle{run vs enter}
	\texttt{ \\
		\# do everything from outside \\
		vcsh run zsh git add .zshrc \\
		vcsh run zsh git commit \\
		vcsh run zsh git push \\
		\# the same, but from within\\
		vcsh enter zsh \\
		git add .zshrc \\
		git commit \\
		git push \\
		exit
	}
\end{frame}

\begin{frame}
	\frametitle{Playing nice with others}
	\begin{itemize}
		\item shells can display exported \texttt{ENV} in \texttt{\$PROMPT}
		\item vcs\_info
		\item mr via plugin, mainline soon
		\item git-annex to manage non-configuration files
		\item Simple but powerful hook system
	\end{itemize}
\end{frame}


\section{Outlook}

\begin{frame}
	\frametitle{Outline}
	\tableofcontents[currentsection]
\end{frame}

\begin{frame}
	\frametitle{Future work}
	\begin{itemize}
		\item More unit tests
		\item Get vcsh into more distributions
		\item Spread awareness to reach critical mass
		\item Maybe extend support to subversion, mercurial, etc
	\end{itemize}
\end{frame}

\section{Outro}

\begin{frame}
	\frametitle{Outline}
	\tableofcontents[currentsection]
\end{frame}

\begin{frame}
	\frametitle{Where to get it}
	\begin{itemize}
		\item \texttt{git clone git://github.com/RichiH/vcsh.git}
		\item Native packages for
		\begin{itemize}
			\item Debian
			\item Ubuntu
			\item Arch Linux (AUR)
		\end{itemize}
		\item Small bug in README.md, use v0.20120203 or git
	\end{itemize}
\end{frame}

\begin{frame}
	\frametitle{Thanks!}
		\begin{center}
			\vfill
			Thanks for listening!\\
			\vfill
			Questions? Follow me outside when my time-slot is over.
			\vfill
			See slide footer for further contact Information.
			\vfill
		\end{center}
\end{frame}




\end{document}


%\begin{frame}
%	\frametitle{}
%	\begin{itemize}
%		\item 
%		\item 
%		\item 
%		\item 
%		\item 
%	\end{itemize}
%\end{frame}
