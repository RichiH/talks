\documentclass[t]{beamer}
\usepackage{helvet}
\usepackage{calc}
\usepackage[utf8]{inputenc}
\usepackage[english]{babel}

\usetheme{Ilmenau}

\setbeamercovered{transparent}
\setbeamertemplate{navigation symbols}{}

\usepackage{units}
\usepackage{amsbsy}
\usepackage{amsmath}
\usepackage{amssymb}
\usepackage{graphics}
\usepackage{graphicx}
\usepackage{epsf}
\usepackage{epsfig}
\usepackage{fixmath}
%\usepackage{pgfmath}
\usepackage{wrapfig}


\title{git-annex}
\subtitle{manage files with git, without checking their contents in}
\author{Richard Hartmann,\\
RichiH@\{freenode,OFTC,IRCnet\},\\
richih.mailinglist@gmail.com}
\date{2012-02-05}



\begin{document}


\begin{frame}
	\titlepage
\end{frame}

\begin{frame}
	\frametitle{Outline}
	\tableofcontents
\end{frame}


\section{Intro}

\begin{frame}
	\frametitle{Outline}
	\tableofcontents[currentsection]
\end{frame}

\begin{frame}
	\frametitle{Who am I?}
	\begin{itemize}
		\item Project \& Network Operations Manager at Globalways AG
		\item freenode \& OFTC staff
		\item Passionate about FLOSS
		\item I am not the author of git-annex, but an interested early adopter
	\end{itemize}
\end{frame}

\begin{frame}
	\frametitle{What is git?}
	\begin{itemize}
		\item Version control system
		\item Distributed
		\begin{itemize}
			\item Does not need a central repository
			\item Allows you to commit while offline
		\end{itemize}
		\item Requires you to have the \textbf{complete} history of all files in every checkout
		\item Currently the best version control system available (imo...)
		\item (git using SHA1 was a stupid idea from day one)
	\end{itemize}
\end{frame}

\begin{frame}
	\frametitle{What is git-annex?}
	\begin{itemize}
		\item Based on git
		\item Does not need to check files into git
		\item Still allows you to check files into git if you want to
		\item Able to maintain full history, but does not do so by default
		\item Enables various workflows
	\end{itemize}
\end{frame}


\section{Use cases}

\begin{frame}
	\frametitle{Outline}
	\tableofcontents[currentsection]
\end{frame}

\begin{frame}
	\frametitle{The Archivist}
	\begin{itemize}
		\item Put data into git-annex
		\item Distribute data among any number of drives, tapes, remotes, etc
		\item Store offline media in a save place
		\item Maintain full information about number and location of all copies
	\end{itemize}
\end{frame}

\begin{frame}
	\frametitle{Media consumption}
	\begin{itemize}
		\item Import podcasts, videos, and slides
		\item Sync or export to consumption devices
		\item Consume media
		\item Drop consumed media from annex
		\item Deletion will propagate through all annexes over time
	\end{itemize}
\end{frame}

\begin{frame}
	\frametitle{The Nomad}
	\begin{itemize}
		\item Keep copies of data on the Internet
		\item Optionally sync between several local devices for backup
		\item Add data locally and/or remotely while on the road
		\item Sync data between local and remote once at an Internet cafe or similar
		\item Perfect for photos while travelling!
	\end{itemize}
\end{frame}


\section{Technical details}

\begin{frame}
	\frametitle{Outline}
	\tableofcontents[currentsection]
\end{frame}

\begin{frame}
	\frametitle{Internal workings 1/2}
	\begin{itemize}
		\item Written in Haskell, so strong typing etc internally
		\item Moves files into \texttt{.git/annex/objects}
		\item Makes them read-only (important to how git-annex works)
		\item Puts symlink in place of file
		\item Stores location data ("this file is in annex foo") in branch \texttt{git-annex}
		\item User adds and commits symlinks to normal git repo (usually the master branch)
	\end{itemize}
\end{frame}

\begin{frame}
	\frametitle{Internal workings 2/2}
	\begin{itemize}
		\item Read-only files force you to \texttt{git annex unlock} prior to changing them
		\item Ensures that you will \texttt{git annex add} all unlocked files
		\item git-annex can then discard or keep old data, depending on setup
	\end{itemize}
\end{frame}

\begin{frame}
	\frametitle{Special remotes 1/2}
	\begin{itemize}
		\item Allow you to store data in non-git remotes
		\item git-annex still tracks all data in special remotes(!)
		\item Support encryption for storage on untrusted machines/media
		\item Hook system which lets you write to and read from arbitrary remotes
	\end{itemize}
\end{frame}

\begin{frame}
	\frametitle{Special remotes 2/2}
	\begin{itemize}
		\item Amazon S3 (and compatible, like OpenStack Swift)
		\item bup
		\item directory (basically exports your data)
		\item rsync
		\item Tahoe-LAFS (cloud-based, distributed file system)
		\item web (great for talks, slides, project Gutenberg, or archive.org)
	\end{itemize}
\end{frame}

\begin{frame}
	\frametitle{Data integrity}
	\begin{itemize}
		\item Allows you to set minimal number of required copies per suffix, directory, etc
		\item Uses SHA1, SHA2-\{224,256,384,512\} to ensure integrity
		\item All remotes and special remotes can be verified, as well
		\begin{itemize}
			\item remotes verify locally and transmit the result
			\item special remotes need to transfer all data
		\end{itemize}
		\item Verification takes required amount of copies into account
		\item \texttt{git fsck; git annex fsck}
	\end{itemize}
\end{frame}


\section{Outro}

\begin{frame}
	\frametitle{Outline}
	\tableofcontents[currentsection]
\end{frame}

\begin{frame}
	\frametitle{Where to get it}
	\begin{itemize}
		\item \texttt{cabal install git-annex --bindir=\$HOME/bin}
		\item Native packages for
		\begin{itemize}
			\item Debian
			\item Ubuntu
			\item FreeBSD
			\item Arch Linux
			\item NixOS
		\end{itemize}
	\end{itemize}
\end{frame}

\begin{frame}
	\frametitle{Thanks!}
		\begin{center}
			\vfill
			Thanks for listening!\\
			\vfill
			Questions? Follow me outside when my time-slot is over.
			\vfill
			See slide footer for further contact Information.
			\vfill
		\end{center}
\end{frame}




\end{document}


%\begin{frame}
%	\frametitle{}
%	\begin{itemize}
%		\item 
%		\item 
%		\item 
%		\item 
%		\item 
%	\end{itemize}
%\end{frame}
