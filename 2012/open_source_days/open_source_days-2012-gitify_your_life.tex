\documentclass[t]{beamer}
\usepackage{helvet}
\usepackage{calc}
\usepackage[utf8]{inputenc}
\usepackage[english]{babel}

\usetheme{Ilmenau}

\setbeamercovered{transparent}
\setbeamertemplate{navigation symbols}{}

\usepackage{units}
\usepackage{amsbsy}
\usepackage{amsmath}
\usepackage{amssymb}
\usepackage{graphics}
\usepackage{graphicx}
\usepackage{epsf}
\usepackage{epsfig}
\usepackage{fixmath}
%\usepackage{pgfmath}
\usepackage{wrapfig}


\title{Gitify your life}
\subtitle{web, blog, configs, data, and backups}
\author{Richard Hartmann,\\
RichiH@\{freenode,OFTC,IRCnet\},\\
richih.mailinglist@gmail.com}
\date{2012-03-11}



\begin{document}

% hide all subsections
\setcounter{tocdepth}{1}

\begin{frame}
	\titlepage
\end{frame}

\begin{frame}
	\frametitle{Outline}
	\tableofcontents
\end{frame}


\section{Intro}

\begin{frame}
	\frametitle{Outline}
	\tableofcontents[currentsection]
\end{frame}

\subsection{Personal stuff}

\begin{frame}
	\frametitle{Who am I?}
	\begin{itemize}
		\item Richard "RichiH" Hartmann
		\item freenode \& OFTC staff
		\item Passionate about FLOSS
		\item Author of vcsh
	\end{itemize}
\end{frame}

\subsection{The basics}

\begin{frame}
	\frametitle{What is git?}
	\begin{itemize}
		\item Version control system
		\item Distributed
		\begin{itemize}
			\item No need for central repository
			\item Allows you to commit while offline
		\end{itemize}
		\item Full history in every checkout
		\item Best version control system available (imo...;)
	\end{itemize}
\end{frame}


\section{ikiwiki}

\begin{frame}
	\frametitle{Outline}
	\tableofcontents[currentsection]
\end{frame}

\subsection{Background}

\begin{frame}
	\frametitle{What is ikiwiki?}
	\begin{itemize}
		\item Written in Perl
		\item Can use git or subversion as back-end
		\item Offers web-based editing and CLI push/pull
		\item Parses various markup languages
		\item Extensive templating and CSS support
		\item Acts as Wiki, CMS, and blog
		\item RSS and Atom feed for whole site, per page, per tag, etc
		\item Supports OpenID
	\end{itemize}
\end{frame}

\begin{frame}
	\frametitle{Supported markup languages}
	\begin{itemize}
		\item MarkDown, extended to support
		\begin{itemize}
			\item WikiLink ([[LinkToArticle]])
			\item directive ([[!tag talk/gitify]], [[!author RichiH]], etc)
		\end{itemize}
		\item WikiText
		\item HTML
		\item reStructuredText
		\item Textile
	\end{itemize}
\end{frame}

\begin{frame}
	\frametitle{How does it work?}
	\begin{itemize}
		\item Save data into repository
		\item Triggered by commit hook or web commit
		\item Parses input files
		\item Compiles into HTML, create new pages, updates RSS, etc
		\item Commits MarkDown source for autocreated/-changed pages into repository
		\item User can then pull changes to local repository
	\end{itemize}
\end{frame}

\subsection{Use cases}

\begin{frame}
	\frametitle{Common uses}
	\begin{itemize}
		\item Public Wiki
		\item Private notes
		\item Blog
		\item CMS
	\end{itemize}
\end{frame}

\begin{frame}
	\frametitle{Adding/editing content}
	\begin{itemize}
		\item Web-based (useful, but boring)
		\item CLI-based (awesome!)
		\item GUI-based
	\end{itemize}
\end{frame}

\begin{frame}
	\frametitle{Advanced usage}
	\begin{itemize}
		\item Interface with source files, only
		\item Maintain wiki and docs in the same repository as your source code
		\item Separate staging or even preview branches
	\end{itemize}
\end{frame}


\section{etckeeper}

\begin{frame}
	\frametitle{Outline}
	\tableofcontents[currentsection]
\end{frame}

\subsection{Minimal, quick overview}

\begin{frame}
	\frametitle{In a word}
	\begin{itemize}
		\item Auto-commits /etc prior to and after all actions by packet manager
		\item Hooks into apt, yum, and cron
		\item Allows manual commits
		\item Various back-ends
		\begin{itemize}
			\item bzr
			\item darcs
			\item git
			\item mercurial
		\end{itemize}
		\item Easy way to recover from failures, misconfiguration or to clone machines
	\end{itemize}
\end{frame}


\section{vcsh}
\begin{frame}
	\frametitle{Outline}
	\tableofcontents[currentsection]
\end{frame}

\subsection{Technical details}

\begin{frame}
	\frametitle{What is vcsh?}
	\begin{itemize}
		\item Implemented in POSIX shell; portable
		\item "version control shell" or "version control system \texttt{\$HOME}"
		\item Based on git
		\begin{itemize}
			\item git unable to maintain several working copies in one directory
			\item Sucks if you want to keep your configs in git
		\end{itemize}
		\item vcsh uses fake bare git repositories to work around this
		\item Think of it as an extension to git
	\end{itemize}
\end{frame}

\begin{frame}
	\frametitle{fake bare.. what?}
	\begin{itemize}
		\item Normal git repo:
		\begin{itemize}
			\item working copy in \texttt{\$GIT\_WORK\_TREE}
			\item git data in \texttt{\$GIT\_WORK\_TREE/.git} aka \texttt{\$GIT\_DIR}
		\end{itemize}
		\item Bare git repo:
		\begin{itemize}
			\item git data in \texttt{\$GIT\_DIR}
			\item no  \texttt{\$GIT\_WORK\_TREE}
		\end{itemize}
		\item Fake bare git repo:
		\begin{itemize}
			\item working copy in \texttt{\$GIT\_WORK\_TREE}
			\item git data in \texttt{\$GIT\_DIR}
			\item \texttt{\$GIT\_WORK\_TREE == \$HOME}
			\item \texttt{\$GIT\_DIR == \$XDG\_CONFIG\_HOME/vcsh/repo.d/\$repo.vcsh}
			\item \texttt{core.bare = false}
		\end{itemize}
	\end{itemize}
\end{frame}

\begin{frame}
	\frametitle{Problems with fake bare git repos}
	\begin{itemize}
		\item Fake bare repos are messy to set up and use
		\item Reason why git disallows shared \texttt{\$GIT\_WORK\_TREE}: complexity due to context-dependency
		\item Mistakes lead to confusion or data loss; imagine \texttt{\$GIT\_WORK\_TREE} set and
		\begin{itemize}
			\item \texttt{git add}
			\item \texttt{git reset --hard HEAD\~{}1}
			\item \texttt{git checkout -- *}
			\item \texttt{git clean -f}
		\end{itemize}
	\end{itemize}
\end{frame}

\begin{frame}
	\frametitle{Solution: vcsh}
	\begin{itemize}
		\item Wraps around git
		\item Hides complexity and does sanity checks
			\item Several git repos checked out into \texttt{\$HOME} at once
		\begin{itemize}
			\item One repo for zsh, Vim, mplayer, etc
			\item Enables specific subsets of repos per host
		\end{itemize}
		\item Manages complete repo life-cycle
	\end{itemize}
\end{frame}

\begin{frame}
	\frametitle{Outline}
	\tableofcontents[currentsection]
\end{frame}

\subsection{Using vcsh}

\begin{frame}
	\frametitle{Create new repo}
	\texttt{ \\
		\# create new repo \\
		vcsh init vim \\
		\# add files to it \\
		vcsh run vim git add .vim .vimrc \\
		\# commit using shorthand form \\
		vcsh vim commit \\
		\# push using longhand form \\
		vcsh run vim git push
	}
\end{frame}

\begin{frame}
	\frametitle{Made-up life-cycle}
	\texttt{ \\
		\# clone repo into new name zsh \\
		vcsh clone git://github.com/RichiH/zshrc.git zsh \\
		\# optionally update legacy repos \\
		vcsh setup zsh \\
		\# display all files managed by this repo \\
		vcsh run zsh git ls-files \\
		\# rename repo just because \\
		vcsh rename zsh zshrc \\
		\# delete repo \\
		vcsh delete zshrc
	}
\end{frame}

\begin{frame}
	\frametitle{run vs enter}
	\texttt{ \\
		\# do everything from outside \\
		vcsh run zsh git add .zshrc \\
		vcsh run zsh git commit \\
		vcsh run zsh git push \\
		\# the same, but from within\\
		vcsh enter zsh \\
		git add .zshrc \\
		git commit \\
		git push \\
		exit
	}
\end{frame}

\begin{frame}
	\frametitle{Playing nice with others}
	\begin{itemize}
		\item shells can display exported \texttt{ENV} in \texttt{\$PROMPT}
		\item vcs\_info
		\item mr via plugin, mainline soon
		\item git-annex to manage non-configuration files
		\item Simple but powerful hook system
	\end{itemize}
\end{frame}


\section{git-annex}

\begin{frame}
	\frametitle{Outline}
	\tableofcontents[currentsection]
\end{frame}

\subsection{Background}

\begin{frame}
	\frametitle{What is git-annex?}
	\begin{itemize}
		\item Based on git
		\item No need to check files into git
		\item Still able to check files into git if you want
		\item Able to maintain complete data history; does not do so by default
		\item Written with low bandwidth and flaky connections in mind
		\item Various work-flows
	\end{itemize}
\end{frame}

\begin{frame}
	\frametitle{Internal workings 1/2}
	\begin{itemize}
		\item Written in Haskell, so strong typing etc, internally
		\item Uses rsync to transfer data
		\item Moves files into \texttt{.git/annex/objects}
		\item Makes files read-only
		\item Stores location data in branch \texttt{git-annex}
		\item Puts symlink in place of file
		\item User adds and commits symlinks to master branch
	\end{itemize}
\end{frame}

\begin{frame}
	\frametitle{Internal workings 2/2}
	\begin{itemize}
		\item Read-only files force you to \texttt{git annex unlock} prior to changing them
		\item Ensures that you will \texttt{git annex add} all unlocked files
		\item git-annex can then discard or keep old data, depending on setup
	\end{itemize}
\end{frame}

\begin{frame}
	\frametitle{Data integrity}
	\begin{itemize}
		\item Set minimal number of required copies per suffix, directory, etc
		\item SHA1, SHA2-\{224,256,384,512\} for integrity
		\item All remotes and special remotes can be verified
		\begin{itemize}
			\item remotes verify locally and transmit the result
			\item special remotes transfer all data to verify
		\end{itemize}
		\item Verification takes required amount of copies into account
		\item \texttt{git fsck; git annex fsck}
	\end{itemize}
\end{frame}

\begin{frame}
	\frametitle{Special remotes 1/2}
	\begin{itemize}
		\item Stores data in non-git-annex remotes
		\item Tracks all data stored in special remotes
		\item Supports encryption for storage on untrusted machines/media
		\item Hook system lets you write to and read from arbitrary remotes
	\end{itemize}
\end{frame}

\begin{frame}
	\frametitle{Special remotes 2/2}
	\begin{itemize}
		\item bup
		\item directory
		\item rsync
		\item S3, Swift, etc
		\item Tahoe-LAFS
		\item web (media.ccc.de, Project Gutenberg, archive.org, etc)
	\end{itemize}
\end{frame}


\subsection{Use cases}

\begin{frame}
	\frametitle{The Archivist}
	\begin{itemize}
		\item Put data into git-annex
		\item Distribute data among any number of drives, tapes, remotes, etc
		\item Store offline media in a safe place
		\item Maintain full information about number and location of all copies
	\end{itemize}
\end{frame}

\begin{frame}
	\frametitle{Media consumption}
	\begin{itemize}
		\item Import podcasts, videos, and slides
		\item Sync or export to consumption devices
		\item Consume media
		\item Drop consumed media from annex
		\item Deletion propagates through all annexes over time
	\end{itemize}
\end{frame}

\begin{frame}
	\frametitle{The Nomad}
	\begin{itemize}
		\item Keep copies of data on www
		\item Optionally sync between several local devices for backup
		\item Add data locally and/or remotely while on the road
		\item Sync data between local and remote once at an Internet café or similar
		\item Perfect for photos while travelling
	\end{itemize}
\end{frame}


\section{bup}

\begin{frame}
	\frametitle{Outline}
	\tableofcontents[currentsection]
\end{frame}

\subsection{One-slide-overview}

\begin{frame}
	\frametitle{In a word...}
	\begin{itemize}
		\item Written in Python
		\item Fast
		\item Very space-efficient (reduced 120 GiB (rsnapshot) to 45 GiB)
		\item Built-in de-duplication
		\item Can be mounted as FUSE
		\item Can not drop old data (yet)
	\end{itemize}
\end{frame}

\section{mr}

\begin{frame}
	\frametitle{Outline}
	\tableofcontents[currentsection]
\end{frame}

% sync all my digital life in five minutes
\subsection{Too many repos...}

\begin{frame}
	\frametitle{Tying it all together}
	\begin{itemize}
		\item 
		\item 
		\item 
		\item 
		\item 
	\end{itemize}
\end{frame}


\section{Outro}

\begin{frame}
	\frametitle{Outline}
	\tableofcontents[currentsection]
\end{frame}

\begin{frame}
	\frametitle{Resources}
	Most of these are packaged for the major distributions
	\begin{itemize}
		\item ikiwiki: \url{http://ikiwiki.info/}
		\item etckeeper: \url{http://joey.kitenet.net/code/etckeeper/}
		\item vcsh: \url{https://github.com/RichiH/vcsh}
		\item git-annex: \url{http://git-annex.branchable.com/}
		\item bup: \url{https://github.com/apenwarr/bup}
		\item mr: \url{http://kitenet.net/~joey/code/mr/}
	\end{itemize}
\end{frame}

\begin{frame}
	\frametitle{Previous talks}
	Previous talks, a bit more in-depth than this one:
	\begin{itemize}
		\item vcsh: \url{http://fosdem.org/2012/schedule/event/vcsh}
		\item git-annex: \url{http://fosdem.org/2012/schedule/event/gitannex}
	\end{itemize}
\end{frame}


\begin{frame}
	\frametitle{Thanks!}
		\begin{center}
			\vfill
			Thanks for listening!\\
			\vfill
			Questions? Ask now or follow me outside once my time-slot is over.
			\vfill
			See slide footer for further contact information.\\
			\vfill
			\#vcs-home @ irc.oftc.net\\
			vcs-home@lists.madduck.net
			\vfill
		\end{center}
\end{frame}


\end{document}


%\begin{frame}
%	\frametitle{}
%	\begin{itemize}
%		\item 
%		\item 
%		\item 
%		\item 
%		\item 
%	\end{itemize}
%\end{frame}
