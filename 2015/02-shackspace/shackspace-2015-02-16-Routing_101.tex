\documentclass[t]{beamer}
\usepackage{helvet}
\usepackage{calc}
\usepackage[utf8]{inputenc}
\usepackage[english]{babel}

\usetheme{Ilmenau}

\setbeamercovered{transparent}
\setbeamertemplate{navigation symbols}{}

\usepackage{units}
\usepackage{amsbsy}
\usepackage{amsmath}
\usepackage{amssymb}
\usepackage{graphics}
\usepackage{graphicx}
\usepackage{epsf}
\usepackage{epsfig}
\usepackage{fixmath}
%\usepackage{pgfmath}
\usepackage{wrapfig}


\title{Routing 101}
\subtitle{Eine Einf\"uhrung}
\author{Richard Hartmann,\\
RichiH@\{freenode,OFTC,IRCnet\},\\
richih@debian.org}
\date{2015-02-16}



\begin{document}

% hide all subsections
\setcounter{tocdepth}{1}

\begin{frame}
	\titlepage
\end{frame}

\begin{frame}
	\frametitle{Outline}
	\tableofcontents
\end{frame}


\subsection{Personal stuff}

\begin{frame}
	\frametitle{Who am I?}
	\begin{itemize}
		\item Richard "RichiH" Hartmann
		\item Backbone and project manager at Globalways AG
		\item freenode \& OFTC staff
		\item FOSDEM, DebConf, DENOGx staff
		\item Debian Developer
		\item Author of vcsh
%		\item Passionate about FLOSS
	\end{itemize}
\end{frame}

\section{Intro}

\begin{frame}
	\frametitle{Outline}
	\tableofcontents[currentsection]
\end{frame}

\subsection{ISO/OSI}

\begin{frame}
	\frametitle{Layer 1}
	\begin{itemize}
		\item Physical Layer
		\item Kabel
		\item Kupfer, Wireless, Glas, etc
	\end{itemize}
\end{frame}

\begin{frame}
	\frametitle{Layer 2}
	\begin{itemize}
		\item Data Link Layer
		\item "Lokales Netzwerk"
		\item Jeder kann mit jedem reden
		\item Einfaches Setup
		\item ...warum mehr?
	\end{itemize}
\end{frame}

\begin{frame}
	\frametitle{Layer 2}
	\begin{itemize}
		\item ...weil Ethernet nicht skaliert
		\item Broadcasts
		\item ARP
		\item Loops (Spanning Tree)
	\end{itemize}
\end{frame}

\begin{frame}
	\frametitle{Layer 3}
	\begin{itemize}
		\item Network Layer
		\item Hier: IP
		\item Kleinster gemeinsamer Nenner im Internet
		\item Skaliert
		\item Versteckt die meiste Komplexit\"at
		\item Heutiger Fokus
	\end{itemize}
\end{frame}

\begin{frame}
	\frametitle{Layer 4-7}
	\begin{itemize}
		\item Layer 4: Transport Layer (TCP, UDP, etc)
		\item Layer 5: Session Layer (hier: Sockets, etc)
		\item Layer 6: Presentation Layer (hier: MIME, TLS, etc)
		\item Layer 7: Application Layer (hier: HTTP, DNS, SMTP, etc)
	\end{itemize}
\end{frame}

\section{Statisches Routing}

\begin{frame}
	\frametitle{Outline}
	\tableofcontents[currentsection]
\end{frame}

\subsection{Laaaangweilig...}

\begin{frame}
	\frametitle{LAN}
	\begin{itemize}
		\item 192.168.0.x \& 255.255.255.0
		\item CIDR: 192.168.0.x/24
		\item Nur lokales Routing
		\item Ziemlich langweilig
	\end{itemize}
\end{frame}

\begin{frame}
	\frametitle{WAN}
	\begin{itemize}
		\item "Wohin mit meinem Traffic?"
		\item Meistens: Default Gateway
		\item 0.0.0.0/0 \& ::/0
		\item Optional auch andere more specific Routen
		\item Alles (meist) von Hand
		\item Keine Ausfallsicherheit
	\end{itemize}
\end{frame}


\section{Dynamisches Routing}

\begin{frame}
	\frametitle{Outline}
	\tableofcontents[currentsection]
\end{frame}

\subsection{\ }

\begin{frame}
	\frametitle{Grundlagen}
	\begin{itemize}
		\item "Das Netz" kann selbst entscheiden wo welcher Traffic entlang fliesst
		\item Ausfallsicherheit
		\item Automatisches Re-Routing
		\item Darauf basiert das Internet
	\end{itemize}
\end{frame}

\begin{frame}
	\frametitle{OSPF}
	\begin{itemize}
		\item Open Shortest Path First
		\item Interior Gateway Protocol
		\item Schnell
		\item Alle Peers vertrauen einander (fast) komplett
		\item Skaliert bedingt (Areas k\"onnen helfen)
		\item Meistens gut genug
		\item F\"ur echte Skalierung gleich IS-IS verwenden
	\end{itemize}
\end{frame}

\begin{frame}
	\frametitle{BGP}
	\begin{itemize}
		\item Border Gateway Protocol
		\item intern (IBGP) \& extern (EBGP) zu verwenden
		\item Skaliert de facto beliebig
		\item Ist erweiterbar; Implementierungen ignorieren was sie nicht verstehen
		\item Filter etc einfach zu implementiern
		\item Signalisierung via Communities mit eingebaut
		\item Alle ASe verwenden BGP um miteinander Routen auszutauschen
	\end{itemize}
\end{frame}

\begin{frame}
	\frametitle{AS was?}
	\begin{itemize}
		\item AS == Postleitzahl
		\item Subnetz == Strasse
		\item Ip Adresse == Hausnummer
		\item Port == Wohnung/Klingelschild
	\end{itemize}
\end{frame}


\section{Internet}

\subsection{\ }

\begin{frame}
	\frametitle{Inter-AS Kommunikation}
	\begin{itemize}
		\item Upstream
		\item Downstream
		\item Peering
	\end{itemize}
\end{frame}

\begin{frame}
	\frametitle{Route-Objects}
	\begin{itemize}
		\item Wie k\"onnen sich die Teilnehmer vertrauen?
		\begin{itemize}
			\item Tun sie nicht, sie filtern
		\end{itemize}
		\item In der RIPE Region einigermassen sauber hinterlegt; Rest der Welt Chaos
		\item Heute hat der shack: 93.231.162.148
		\item route: 93.192.0.0/10AS3320
		\item Die Route 93.192.0.0/10 darf von AS 3320 kommen
	\end{itemize}
\end{frame}

\begin{frame}
	\frametitle{Probleme 1/2}
	\begin{itemize}
		\item Konvergenzzeiten
		\item Routing-Loops
		\item Schlechte/falsche Filter
	\end{itemize}
\end{frame}

\begin{frame}
	\frametitle{Probleme 2/2: Route-Hijacking}
	\begin{itemize}
		\item FOSDEM15 hatte 151.216.128.0/17
		\item 151.216.128.0/18 \& 151.216.192.0/18 wurden aus Russland von AS28840 announced
		\item L\"osung: 151.216.128.0/19, 151.216.160.0/19, 151.216.192.0/19 \& 151.216.224.0/19 announced
		\item YouTube Block 2008 in Pakistan
		\item Regelm\"assige Hijacks aus China
		\item Unter anderem darum End-to-End-Security!
	\end{itemize}
\end{frame}


\section{Outro}

\subsection{The End!}

\begin{frame}
	\frametitle{Thanks!}
		\begin{center}
			\vfill
			Thank you for listening!\\
			\vfill
			Questions? Ask now or catch me after this talk, both is fine.
			\vfill
			See slide footer for further contact information.\\
			\vfill
			\url{http://richardhartmann.de/talks/}
			\vfill
		\end{center}
\end{frame}


\end{document}


%\begin{frame}
%	\frametitle{}
%	\begin{itemize}
%		\item 
%		\item 
%		\item 
%		\item 
%		\item 
%	\end{itemize}
%\end{frame}
